\chapter{Especificação de Requisitos do Sistema}\label{chap:especificacao-requisitos-sistema}
Neste capítulo, será abordado um pouco melhor sobre os requisitos do sistema encontrados, os pontos levantados nas diversas reuniões e os requisitos funcionais e não funcionais encontrados. Vale lembrar que, no apêndice \href{chap:vision-doc-appendix}, há maiores detalhes sobre o resultado geral do processo de levantamento de requisitos.

\section{Pontos Levantados nas Reuniões}
Esta seção aborda os diversos pontos levantados nas reuniões, as funcionalidades gerais encontradas e como elas formaram os casos de uso do sistema.

Como principal foco do sistema, tanto o Prof. Dr. João Batista como o Prof. Dr. Pauo Cugnasca querem automatizar o processo atual e permitir seu crescimento, em especial com a participação das empresas, aproveitando a tendência de virutalização.

Algumas funcionalidades gerais levantadas nas reuniões com os \textit{stakeholders}:

\begin{itemize}
    \item Integrar participação do orientador no processo: Facilitar comunicação orientador/coordenadores e permitir acompanhamento mais frequente do orientador.
    \item Base de monografias antigas públicas: Permitir busca e filtro com base em parâmetros.
    \item Incluir empresas para acompanhar monografias: Permitir acesso à monografia antes para realizar avaliação.
    \item Automatizar processo de avaliação: Permitir avaliação tanto da banca como da feira (com as empresas).
    \item Realizar match de temas e orientadores e ideação de temas: Permitir que orientadores, alunos e empresas proponham temas e
    \item combinar grupos de alunos ou alunos individuais e orientadores para realizar trabalhos.
    \item Controle de recursos: Permitir ao Nilton gerenciar recursos necessários para as apresentações (imprimir apresentações sem necessidade do aluno entregar arquivos via pen drive, por exemplo).
    \item Mobilidade: Permitir que avaliações sejam feitas inclusive mobile.
    \item Construção de bancas: Permitir construção de bancas de TCC, para avaliação.
    \item Convite em massa (mala direta): Mala direta para convidar pessoas.
    \item Confiabilidade: resistir a situações adversas (backup de dados).
    \item Hospedagem: deve ser interna, no servidor do PCS USP.
\end{itemize}

Além das funcionalidades gerais, também foi levantado o processo atual das disciplinas, resumido nos seguintes tópicos:
\begin{itemize}
    \item Pré-disciplinas: Conversa no quarto ano para orientar alunos a pensarem nos temas.
    \item Estrutura geral das duas disciplinas: Reuniões intermediárias de acompanhamento, com entregas (apresentações usadas e documentação preliminar correspondente).
    \item Na disciplina de TCC2: Processo de avaliação com Feira e Banca.
    \begin{itemize}
        \item Formação e divulgação de bancas: escolha dos participantes.
        \item Realização da Feira e da Defesa perante banca.
        \item Correção do TCC, com a validação do orientador.
    \end{itemize}
\end{itemize}

\section{Requisitos Funcionais}
Dadas as funcionalidades gerais encontradas, foram levantados os seguintes requisitos funcionais:
\begin{itemize}
    \item Facilitar comunicação entre orientador, coordenadores e alunos.
    \item Buscar as monografias antigas para consulta pública.
    \item Incluir avaliadores da feira para acompanhar monografias correntes.
    \item Permitir avaliação tanto da banca como da feira, permitindo acesso prévio ao conteúdo e facilitando a avaliação.
    \item Permitir que orientadores, alunos e empresas proponham temas e consigam combinar grupos para realizar as propostas.
    \item Permitir aos técnicos gerenciarem recursos necessários para as apresentações.
    \item Permitir a montagem de bancas de TCC.
\end{itemize}

\section{Requisitos Não Funcionais}
Além dos requisitos funcionais, alguns requisitos não funcionais importantes foram levantados aqui:

\begin{itemize}
    \item Confiabilidade: O sistema deve permanecer funcional durante as avaliações finais do curso, pois são cruciais para o bom andamento da matéria.
    \item Segurança: Os acessos às monografias em andamento devem ser apenas aos envolvidos diretos do trabalho. Além disso, as empresas só podem acessar o resultado final não revisado, para fins de avaliação. Foco especial nos requisitos de Confidencialidade e Integridade.
    \item Confiabilidade: O sistema deve suportar situações de falha e lidar bem com as informações, garantindo sua proteção. No caso, foco especial em Recuperabilidade.
    \item Usabilidade: O sistema deve ser bem intuitivo e de aprendizagem fácil, pelo tempo curto dos envolvidos na feira e pela mobilidade envolvida (avaliações pelo celular, por exemplo). Foco especial na Operacionalidade, Estética/Interface e Proteção contra Erros do Usuário.
    \item Manutenibilidade: O sistema deve ser bem intuitivo de realizar alterações, instalações e afins, dado que alunos podem tocar sua manutenção futuramente. Foco especial na Modificabilidade.
\end{itemize}