\chapter{Aspectos Conceituais}\label{chap:aspectos-conceituais}
Neste capítulo, serão abordados os principais conceitos relacionados à parte teórica do trabalho: modelagem de processos, representação dos requisitos envolvidos, a engenharia por detrás desses requisitos, suas descrições por meio de casos de uso e histórias de usuário e os conceitos de verificação e validação.

\section{Modelagem de Processos}
Para entender melhor como o sistema pode atuar em um processo, é valido compreender melhor como funciona esse processo e estabelecer as possíveis fontes de participação do sistema. Esta seção discute um pouco sobre a filosofia de modelar processos e um padrão usado para a modelagem.

\subsection{Filosofia}
A modelagem de processos de negócio (ou o \textit{Business Process Modeling} - BPM) existe para entender melhor como está estabelecido o processo como um todo atuando em diversos aspectos: torna o processo mais fluido; evidencia os pontos de aumento da produtividade dos recursos humanos; aloca problemas para as pessoas corretas e; simplifica as regularizações necessárias no processo\cite{mikehavey2015}.

Há duas abordagens para modelagem de um processo, para entender como o sistema atuará\cite{jessicaangeli2018}:

\begin{itemize}
    \item Antes do sistema (\textit{AS IS}): Modelar o processo antes do sistema, para entender como ele atuará para automatizá-lo.
    \item Junto com o sistema (\textit{TO BE}): Modelar o processo com o sistema atuante, para evidenciar seus impactos.
\end{itemize}

Existem ferramentas para representar os processos de negócio, segundo o CBOK \cite[cap. ~3, p. 79]{cbok2013}:

\begin{itemize}
    \item BPMN (\textit{Business Process Model and Notation})
    \item Fluxograma
    \item EPC (\textit{Event-driven Process Chain})
    \item UML (\textit{Unified Modeling Language})
    \item IDEF (\textit{Integrated Definition Language})
    \item \textit{Value Stream Mapping}
\end{itemize}

\subsection{\textit{Business Process Model and Notation} - BPMN}
Segundo a \citeauthor{bpmn2013}: 

\begin{citacaoLonga}
"O objetivo do BPMN é fornecer uma notação que seja prontamente compreensível por todos os usuários de negócios, desde os analistas que criam os rascunhos iniciais dos processos, até os desenvolvedores técnicos responsáveis pela implementação da tecnologia que executará esses processos e, finalmente, para os empresários que irão gerenciar e monitorar os processos."
\end{citacaoLonga}

Ou seja, o propósito do BPMN é unificar a visão dos processos de negócio usando uma notação universal, compreendida por todos os envolvidos que desejam automatizar etapas deste processo.

Segundo a \citeauthor{bpmn2013}, cap. 7, p. 25, o BPMN possui cinco elementos básicos:

\begin{itemize}
    \item Objeto de Fluxo: Descrevem o comportamento do negócio.
    \item Dados: As informações usadas no negócio.
    \item Conectores: Conectam os objetos com outro objetos e/ou informações.
    \item Piscinas e Raias: Representam, graficamente, um envolvido no processo.
    \item Artefatos: Fornecem informações adicionais sobre os processos.
\end{itemize}

\section{Representação de Requisitos}
Ao se desenvolver um sistema, existem diversas técnicas que podem ser usadas para representar os requisitos. Para este trabalho, duas técnicas foram consideradas: Histórias de Usuário e Casos de Uso.

\subsection{Histórias de Usuário}
Uma das maneiras de representar os requisitos é a História de Usuário (ou \textit{User Story}). Segundo \citeauthor{jonathanrasmusson}:

\begin{citacaoLonga}
Descrições curtas das funcionalidades que o nosso cliente
gostaria de um dia ver em seu software. Elas geralmente são escritas em pequenos cartões de índice (para nos lembrar de não tentar escrever tudo) e estão lá para nos encorajar a ir falar com nossos clientes.
\end{citacaoLonga}

O essencial em escrever boas histórias de usuário está em agregar valor para os \textit{stakeholders}. Ou seja, escrever histórias de usuário que tangem assuntos como arquitetura do sistema, linguagem de implementação, padrões de código entre outros não são boas histórias de usuário, pois salvo exceções, o cliente não vê valor em histórias desse tipo\cite{jonathanrasmusson}.

Em contrapartida, histórias sobre o comportamento do sistema com valor de produto nelas são histórias de usuário importantes para o processo de desenvolvimento. Muitas vezes, porém, histórias de usuário forçam um viés altamente técnico; nessas situações, cabe buscar a causa raiz que levou à "solução" escrita na história: histórias de usuário apresentam problemas de negócio, e não soluções técnicas.

Um padrão para escrever histórias de usuário\cite{jonathanrasmusson}:

\begin{citacaoLonga}
Eu como $<$para quem é a história$>$
\\
Eu quero $<$o que ele quer$>$
\\
por causa $<$por que ele quer$>$
\end{citacaoLonga}

A vantagem de escrever histórias segundo esse modelo é que, naturalmente, elas atendem as características que definem Histórias de Usuário\cite{jonathanrasmusson}:

\begin{itemize}
    \item Independentes: elas podem ser implementadas de maneira isolada e distribuída entre a equipe, permitindo maior facilidade para mudanças no projeto (que ocorrem com frequência).
    \item Negociáveis: podem sofrer alterações e inclusive serem removidas se, com o andamento do projeto, ela perder sua prioridade e tais alterações e remoções não devem afetar o andar geral das outras tarefas.
    \item Testáveis: cada história pode ser verificada testando o fluxo de negócio e com a escrita de testes automatizados, assim garantimos com facilidade a qualidade do que está sendo entregue
    \item Pequenas e Estimáveis: o esforço dela pode ser previsto com antecedência, permitindo assim acompanhar o andamento do projeto.
\end{itemize}

\subsection{Casos de Uso}
Outra alternativa para documentar o que será desenvolvido no sistema são os casos de uso. Essencialmente, cada caso de uso possui um ou mais atores interagindo com o sistema. Segundo \citeauthor{kurtbittnerianspence2002}, \citeyear{kurtbittnerianspence2002}, cap. 1, p. 20: "Os \textbf{atores} representam as pessoas ou coisas que interagem de alguma forma com o sistema. Já os \textbf{casos de uso} representam as coisas de valor que o sistema executa para os atores."

Para escrever casos de uso, foi usada uma estrutura retirada de \citeauthor{ibm2011}, \citeyear{ibm2011}:

\begin{enumerate}
    \item Nome: nome do Caso de Uso
    \begin{enumerate}
        \item Breve Descrição: finalidade do caso de uso
    \end{enumerate}
    \item Fluxo de Eventos: a descrição dos eventos que ocorrem no sistema, que são divididos em dois conjuntos principais.
    \begin{enumerate}
        \item Fluxo Básico: é o comportamento básico da interação entre os atores e o sistema. Não há situações de contorno nem exceções, que serão listadas nos fluxos alternativos.
        \item Fluxos Alternativos: outros comportamentos menos usuais do caso de uso vêm aqui.
    \end{enumerate}
    \item Requisitos Especiais: requisito, não funcional, que não é contemplado no fluxo de eventos.
    \item Pré-condição: estado do sistema antes da realização do caso de uso.
    \item Pós-condição: estado do sistema após a execução.
\end{enumerate}

Resumindo, o caso de uso é uma descrição comportamental da funcionalidade descrita, servindo de contrato estabelecido de desenvolvimento do sistema.

Além da representação textual, há também o uso de representações gráficas de casos de uso. A representação gráfica apresenta o nome do caso de uso e os atores que interagem com ele, mas omite sua parte descritiva. Tais padrões de representação são definidos pela UML\cite{uml2017}.

\section{Requisitos}
Segundo o \citeauthor{ieeecomputersociety2014}: "Um requisito de software é uma propriedade que representa algo para resolver algum problema no mundo real."

Existem dois tipos de requisitos de software, definidos pelo \citeauthor{ieeecomputersociety2014}, cap. 1, p. 3:

\begin{itemize}
    \item Funcionais: Descrevem as funcionalidades executadas pelo software.
    \item Não-funcionais: Definem qualidades do software ou restringem seu funcionamento.
\end{itemize}

Para encontrá-los, existe um exemplo de processo básico, definido em \citeauthor{kurtbittnerianspence2002}:

\begin{itemize}
    \item Definição dos \textit{stakeholders}: Encontrar todos que são afetados pelo sistema.
    \item Entendimento dos Problemas: Técnicas para consultar os \textit{stakeholders} sobre o que eles esperam que o sistema resolva e como sua existência afeta o dia-a-dia deles.
    \item Documento Visão: Unificação do entendimento do sistema, que serve como base para estabelecer o que será de fato o sistema.
    \item Elaboração dos Casos de Uso: Construir os casos de uso do sistema, com base no documento visão levantado.
\end{itemize}

Vale lembrar que, segundo \citeauthor{kurtbittnerianspence2002}, "Todos os requisitos funcionais podem ser capturados como casos de uso, e muitos dos requisitos não funcionais podem ser associados a casos de uso."

Além disso, os casos de uso podem capturar um bom subconjunto dos requisitos funcionais do sistema. Em situações de requisitos ainda não representados pelos casos de uso, eles são capturados por especificações suplementares, definidas pela \citeauthor{ibm2011}\cite[cap. ~1, p. ~23]{kurtbittnerianspence2002}.

\subsection{Definição dos \textit{Stakeholders}}
Inicialmente, para melhor entender quem será afetado com a existência de um sistema, é necessário encontrar quem são essas pessoas ou grupos. Para isso, existe o conceito de \textit{stakeholders}. Segundo \citeauthor{kurtbittnerianspence2002}, cap. 3, p. 88: "Indivíduo que é materialmente afetado pelo resultado do sistema ou o(s) projeto(s) que produzem o sistema."

Ou seja, \textit{stakeholders} não são apenas os indivíduos que efetivamente usarão o sistema, mas sim todos os impactados com sua existência. Dividem-se em:

\begin{itemize}
    \item Usuários: as pessoas que efetivamente usarão o sistema.
    \item Patrocinadores: os financiadores do projeto de software que gerará o sistema.
    \item Desenvolvedores: os responsáveis por desenvolver o sistema.
    \item Autoridades: órgãos reguladores que determinam regras para o uso do sistema.
    \item Consumidores: empresas que compram esses softwares para serem usados.
\end{itemize}

\subsection{Entendimento dos Problemas}
Uma vez mapeado os \textit{stakeholders}, cabe entender as dores que cada um possui e o que eles esperam com o produto final a ser desenvolvido. Alguns processos sugeridos pelo \citeauthor{kurtbittnerianspence2002}:

\begin{itemize}
    \item Entrevistas: entrevistar os \textit{stakeholders} e entender diretamente quais são suas dores e expectativas.
    \item Questionários: sequência de perguntas feitas à todos os \textit{stakeholders} para compreender o que cada um pensa sobre o sistema. São úteis quando há um amplo número de \textit{stakeholders} envolvidos.
    \item Grupo: reunião com alguns representantes dos grupos de \textit{stakeholders} para entender e construir uma visão única sobre o projeto.
    \item Quadros de Aviso: um tipo particular de grupo, com o quadro servindo como unificador da visão, com a diferença de não ter todos reunidos ao mesmo tempo.
    \item \textit{Workshops}: eventos avisados com antecedência para entender melhor sobre o sistema, com a participação dos envolvidos, focando na colaboração entre eles.
    \item Revisões: reuniões informais com o intuito de revisar documentos gerados com alguns envolvidos.
    \item Encenação: uma técnica facilitadora usada em conjunto com \textit{workshops} para obter informações mais específicas ou \textit{feedbacks}.
\end{itemize}

Com essa listagem de problemas levantados, o próximo passo é construir a visão unificada, evidenciando como o software vai atuar no processo. Para isso, há a elaboração de um documento unificando os pontos de vista dos \textit{stakeholders} e estabelecendo o que de fato será o sistema a ser desenvolvido, resultando no Documento Visão.

\subsection{Documento Visão}
O Documento Visão é o responsável por unificar as visões de todos os \textit{stakeholders} do sistema, centralizando todas as expectativas e definindo o escopo macro do projeto. Nele também deve constar as justificativas necessárias para a construção do sistema, explicitando possíveis concorrentes e os diferenciais do produto a ser desenvolvido.

Existem diversos modelos de Documento Visão. Um modelo de referência encontrado em \citeauthor{kurtbittnerianspence2002}, \citeyear{kurtbittnerianspence2002}, cap. 3, p. 135-136:

\begin{enumerate}
    \item Posicionamento: Como o sistema irá se posicionar no quesito de negócios? Há concorrentes que já resolvem o problema? Quais são seus diferenciais em relação a eles?
    \item \textit{Stakeholders} e usuários: Quem são os envolvidos direta e indiretamente com o desenvolvimento e a existência do sistema?
    \item Necessidades chave: Quais são as demandas que realmente precisam estar nos planos do sistema para satisfazer os envolvidos?
    \item Visão geral do produto: O que é o produto de fato? Quais são suas dependências, capacidades e alternativas ao seu desenvolvimento?
    \item Funcionalidades: Quais são as funcionalidades em alto nível do sistema, para que elas resolvam as necessidades chave listadas anteriormente?
    \item Outros requisitos do produto: Quais são os outros requisitos do sistema que não foram capturados como funcionalidades?
\end{enumerate}

\section{Verificação e Validação}
Durante o desenvolvimento, faz parte do processo garantir a qualidade do que está sendo entregue. Para isso, existem os processos de verificação e validação, que segundo o \citeauthor{ieeecomputersociety2014}, \citeyear{ieeecomputersociety2014}, cap. 10, p. 6: "Servem para ajudar a equipe de desenvolvimento a garantir qualidade no sistema por todo seu ciclo de vida."

Ao se desenvolver um software, é uma boa prática acompanhar se os requisitos funcionais e não funcionais estão sendo atendidos, de preferência com acompanhamento das métricas determinadas nos requisitos. Essa prática é conhecida como verificação, definida de maneira resumida pela seguinte pergunta\cite[cap. ~10, p. ~6]{ieeecomputersociety2014}: "Estamos construindo o produto corretamente?"

Além de acompanhar os requisitos propostos, é importante entender se o sistema está de acordo com as expectativas operacionais dos \textit{stakeholders}. Essa análise constante de adequação voltada aos negócios é conhecida como validação, definida pela seguinte questão\cite[cap. ~10, p. ~6]{ieeecomputersociety2014}: "Estamos construindo o produto correto?"

\subsection{Testes}
Em ambos os casos descritos acima, são necessárias técnicas para detectar erros e manter a consistência do que já foi entregue. Para isso, existe o conceito de testes, definido no \citeauthor{ieeecomputersociety2014}, \citeyear{ieeecomputersociety2014}, cap. 4, p. 1: "Consiste na verificação dinâmica de que um programa fornece comportamentos esperados em um conjunto finito de casos de teste, selecionados adequadamente a partir do domínio de execução, geralmente infinito."

Na escala macroscópica, há os testes de aceitação junto aos \textit{stakeholders}, para validar as entregas. Já para testar a arquitetura, costumam ser realizados testes de sistema. Para testar os componentes desenvolvidos do sistema, realizam-se testes de integração entre esses componentes. E, por fim, o código escrito para cada componente elaborado é testado pelos testes de unidade.

Ao desenvolver um sistema, uma boa prática é executar a suíte de testes construída a cada nova funcionalidade adicionada e/ou modificada. Para isso, existe a filosofia de testes automatizados, que são códigos escritos responsáveis por rodar um subconjunto de testes da suíte escolhida. A vantagem dessa filosofia está em reduzir o esforço de trabalho como um todo, dado que é mais custoso refatorar código não-testado problemático do que elaborar as funcionalidades com os testes inclusos\cite[cap. ~3, p. 19-21]{gerardmeszaros2007}.
