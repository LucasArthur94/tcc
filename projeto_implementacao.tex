\chapter{Projeto e Implementação}
Neste capítulo, entraremos em mais detalhes da parte de projeto e implementação do sistema, com base nos requisitos já levantados, passando pelas tecnologias envolvidas e pela arquitetura do sistema.

\section{Tecnologias Utilizadas}
Como parte dos requisitos, era necessário o uso de sistemas \textit{web} para os \textit{stakeholders} terem maior mobilidade e facilidade ao realizarem as interações com o sistema. Sendo assim, é valido passar por alguns conceitos tecnológicos do projeto.

\subsection{\textit{Frameworks web}}

É comum, em desenvolvimento de sistemas, usar soluções prontas como forma de simplificar o desenvolvimento de soluções. Essas soluções são conhecidas como arcabouços (ou \textit{frameworks})\cite{iansommerville2011}: "Um framework é uma estrutura genérica estendida para se criar uma aplicação ou subsistema mais específico.". Para este projeto, foi usado um \textit{framework}, em linguagem Python, chamado \textbf{Django}.

\subsubsection{Django}
O Django é estruturado em aplicações (apps), onde cada aplicação corresponde a uma parte do sistema, geralmente independente e reciclável (ou seja, pode ser usada em aplicações Django diferentes). Cada aplicação segue o padrão \textit{Model-View-Controller}.

O padrão \textit{Model-View-Controller} - MVC é um padrão arquitetural para organizar os componentes da aplicação. Ele é dividido em três grandes grupos\cite{thedjangobook2018}:

\begin{itemize}
    \item Modelo (\textit{Model}): Uma representação (interface) para os dados da aplicação.
    \item Visualização (\textit{View}): Camada de apresentação dos dados da aplicação. No caso do Django, ele é chamado de \textit{Template}.
    \item Controlador (\textit{Controller}): Camada de controle que interliga o modelo com a apresentação dos dados, onde geralmente fica a lógica de negócio. No caso do Django, ele é conhecido como \textit{View}.
\end{itemize}

A vantagem do padrão está no fluxo de dados que existe entre os grupos, além de evitar códigos com funcionalidades diferentes em lugares errados, como, por exemplo, lógica de negócio na camada de apresentação.

\section{Processos de Desenvolvimento}
Aqui vamos tratar um pouco sobre como se deu o processo de desenvolvimento, entendendo melhor os ambientes usados e o fluxo entre eles.

\subsection{Ambientes de Validação}
É uma boa prática, para desenvolvimento de sistemas, criar diversos ambientes de aplicação, para validar as funcionalidades com os diversos \textit{stakeholders}, fora o ambiente onde a aplicação será hospedada, de fato (ou seja, onde ela ficará em ambiente de produção). Cada sistema exige 1 ou mais ambientes durante o fluxo de desenvolvimento, de acordo com a complexidade de negócios do projeto.

\subsubsection{Organização Genérica}
Para o projeto, foram organizados três ambientes de aplicação para realizar os processos de validação e homologação do sistema, para permitir testes isolados dos \textit{stakeholders} sem afetar o fluxo de desenvolvimento. São eles: Desenvolvimento (ou \textit{next-release}), Homologação (ou \textit{staging}) e Produção (ou \textit{production})\cite{tracyragan2017}.

\begin{itemize}
    \item Desenvolvimento: onde geralmente fica a versão mais recente da aplicação, com as novas funcionalidades desenvolvidas e testadas já integradas no ambiente. Essa versão serve como uma prévia do que será entregue para homologação. As mudanças nesse ambiente são mais frequentes, dado que uma nova funcionalidade completa já pode ir para este ambiente.

    \item Homologação: Já neste ambiente, as mudanças são bem menores e servem como ambiente de aprovação dessas mudanças por parte dos \textit{stakeholders}. No caso do sistema de TCCs, ele serviu como homologação com os coordenadores do curso. Se possível, ele deve ser o mais fiél ao ambiente final de produção, assim o comportamento ideal dele em homologação será o mesmo em produção.

    \item Produção: Já este ambiente é onde o sistema irá rodar, de fato. Esse ambiente deve ser o mais estável possível, com alterações apenas homologadas pelos \textit{stakeholders}. Salvo raras exceções, como falhas e problemas encontrados, nada deve ser colocado aqui sem a aprovação no ambiente de homologação.
\end{itemize}

\subsubsection{Estruturação no Projeto}
Para o projeto, foram construídos dois ambientes remotos, fora o ambiente final de produção que deve ficar na nuvem da USP. Os ambientes de desenvolvimento e homologação foram criados no Heroku, um serviço de plataforma (\textit{Platform as a Service} - \textbf{PaaS}), com os seguintes domínios:

\begin{itemize}
    \item Desenvolvimento: \href{https://tccapp-next-release.herokuapp.com}{https://tccapp-next-release.herokuapp.com}
    \item Homologação: \href{https://tccapp-staging.herokuapp.com}{https://tccapp-staging.herokuapp.com}
\end{itemize}

\subsection{Tecnologia Usada: Heroku}

O Heroku é uma plataforma de computação em nuvem conhecida no mercado, com a possibilidade de subir aplicações nas linguagens Ruby, Node.js, Java, Python, Clojure, Scala, Go e PHP. O grande destaque do Heroku está na facilidade em subir uma aplicação com facilidade e de maneira gratuita, o que possibilita testar e validar ideias básicas. A estrutura básica do Heroku funciona com o uso de \textit{dynos}, que são pequenas máquinas tanto para hospedar sua aplicação principal quanto outras auxiliares para serviços externos e/ou paralelos. Resumidamente, ele possui um plano básico gratuito que possibilita testar aplicações de maneira fácil e sem dificuldades de expansão e, se desejar, há máquinas de performance superior, cobradas com base no conceito de \textit{dynos}-hora.

\section{Elaboração e Estrutura do Sistema}
\subsection{Arquitetura}
O Django foi usado na versão mais recente estável (2.1), que corrigiu algumas novidades da versão 2.0. É esta versão que foi usada pelo projeto em questão, dado que é estável e possui suporte de apoio da equipe até Dezembro de 2020\cite{djangodownload}. Ele usa, como padrão, o SQLite, o que atendeu bem durante o desenvolvimento. Já o Heroku tem como banco de dados padrão o PostgreSQL, o que exigiu o chaveamento entre os bancos no ambiente local de desenvolvimento e os ambientes de validação hospedados no serviço. Por isso, o projeto possui configurado dois pacotes de gerenciamento de banco de dados, um para SQLite (nativo do Django), outro para PostgreSQL (o Psycopg\cite{lucassouto2017}).

\section{Problemas Iniciais da Implementação}