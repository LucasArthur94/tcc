\chapter{Teste e Validação}\label{chap:teste-validacao}
Neste capítulo, serão abordadas as etapas de teste e validação do sistema, as reuniões ocorridas e os testes realizados (automatizados ou não).

\section{Validações}
Para validar os Casos de Uso do sistema, algumas reuniões com os \textit{stakeholders} ocorreram durante o processo de desenvolvimento, com o objetivo de acompanhar se o produto estava de acordo com as expectativas levantadas no início do projeto. Nesta seção, será detalhada as reuniões que ocorreram após o fim de cada iteração e as alterações levantadas.

\subsection{Reunião de 28/09}
A reunião foi marcada por ser a primeira validação dos \textit{stakeholders}, logo, o sistema como um todo foi uma completa novidade. Para a reunião, foi construído um roteiro de testes a ser executado em conjunto com os coordenadores da disciplina (Prof. Dr. João Batista e Prof. Dr. Paulo Cugnasca):

\begin{itemize}
    \item Logar como administrador (João ou Paulo)
    \begin{itemize}
        \item Sentir navegação do sistema em todas as partes
        \item Cadastrar 1 aluno
        \item Cadastrar alunos massivamente        
    \end{itemize}

    \item Logar como aluno recém cadastrado
    \begin{itemize}
        \item Sentir navegação do sistema em todas as partes
    \end{itemize}

    \item Logar como administrador (João ou Paulo)
    \begin{itemize}
        \item Cadastrar docente
        \item Cadastrar convidado externo
        \item Cadastrar disciplina
        \item Cadastrar grupo de trabalho
    \end{itemize}

    \item Logar como orientador (docente)
    \begin{itemize}
        \item Confirmar participação
    \end{itemize}

    \item Logar como co-orientador (convidado externo)
    \begin{itemize}
        \item Confirmar participação
    \end{itemize}

    \item Logar como administrador (João ou Paulo)
    \begin{itemize}
        \item Cadastrar atividade
        \item Logar como aluno
        \item Realizar entrega da atividade
    \end{itemize}

    \item Logar como co-orientador
    \begin{itemize}
        \item Revisar atividade
    \end{itemize}

    \item Logar como orientador
    \begin{itemize}
        \item Revisar atividade
    \end{itemize}

    \item Logar como co-orientador
    \begin{itemize}
        \item Tentar revisar atividade
        \item Ver comentários privados
    \end{itemize}
    
    \item Logar como aluno
    \begin{itemize}
        \item Ver comentários públicos
    \end{itemize}
\end{itemize}

Nesta reunião, a aceitação do sistema foi satisfatória, apenas com as seguintes sugestões e ressalvas:

\begin{itemize}
    \item Cadastro massivo de pendentes e inscritos no sistema.
    \item Customização dos textos dos emails.
    \item Flexibilidade nos arquivos das atividades.
    \item Comentários dos alunos nas entregas das atividades.
    \item Alteração/Recuperação de Senha.
\end{itemize}

\subsection{Reunião de 30/11}

Já nesta reunião, o foco da validação era da parte de gerenciamento dos eventos, como alocação de bancas e avaliações em geral. Assim como na primeira reunião, um roteiro de testes foi elaborado para ser usado como base na reunião de validação. Como pré-condição, para facilitar a validação, o ambiente de homologação já terá participantes, grupos, disciplinas, atividades e entregas já presentes no sistema:

\begin{itemize}
    \item Logar como administrador (João ou Paulo)
    \begin{itemize}
        \item Criar evento teórico
        \item Criar evento prático
        \item Alocar grupo para evento teórico
        \item Alocar grupo para evento prático
        \item Avisar convidados por email do evento teórico
        \item Avisar convidados por email do evento prático
    \end{itemize}
    
    \item Logar como convidado da banca teórica (docente)
    \begin{itemize}
        \item Avaliar grupo em qual foi alocado
    \end{itemize}
    
    \item Logar como convidado da banca prática (convidado)
    \begin{itemize}
        \item Avaliar grupo em qual foi alocado
    \end{itemize}
    
    \item Logar como administrador (João ou Paulo)
    \begin{itemize}
        \item Criar fórmula para calcular nota (escolher disciplinas e eventos)
    \end{itemize}
\end{itemize}

Vale ressaltar que, nessa iteração, não foi possível concluir os Casos de Uso relacionados à listagem de entregas e necessidades adicionais.

Como resultado dessa reunião, a aceitação se manteve positiva, sem grandes ressalvas ao resultado.

\section{Testes Executados}
Além do processo de validação, seguindo a filosofia apresentada no capítulo \ref{chap:aspectos-conceituais}, diversos testes foram escritos para reforçar a prática de verificação do sistema, para saber se ele está sendo desenvolvido corretamente. Nesta seção, serão discutidos os cenários de teste escritos e as ferramentas e dificuldades encontradas para escrevê-los.

\subsection{Cenários de Teste}
Dado que o conjunto de testes possíveis é infinito, foram selecionados alguns testes a serem escritos. Para o sistema, dois tipos de teste foram adotados: unidade e interface. 

Nos testes unitários, foram testados apenas os métodos escritos com função diferenciada e primordial para o sistema, como por exemplo os métodos de classe dos modelos. Já os testes de interface cobrem, por meio de ferramentas automatizadas, execuções de tela do sistema, verificando o fluxo de neǵocio de cada aplicação.

Cada aplicação escrita foi testada por uma outra aplicação interna a ela, específica pra teste, contendo os seguintes testes:

\begin{itemize}
    \item Modelo (\textit{test\_model}): Testa os métodos de classe específicos para cada modelo do sistema. Caso não tenha nenhum método escrito, o arquivo fica em branco.
    \item Formulário (\textit{test\_form}): Testa os formulários relacionados à aplicação, o que é útil para verificar se o cadastro de um novo dado está funcionando corretamente.
    \item Tela (\textit{test\_view}): Testa a aplicação como um todo, usando a tela como referência.
\end{itemize}

Todas as telas possuem testes escritos, bem como todos os formulários. Apenas os modelos de alocação, avaliação, usuários e grupos de trabalho não têm testes, dado que não possuem métodos nos modelos. No total, 39 testes foram elaborados para o sistema.

\subsection{Ferramentas de Teste}
O Django possui uma biblioteca extensa para realizar testes no sistema. Para testes unitários, usar a parte nativa resolve o problema. Já para escrever testes de interface, a complexidade foi maior, dado que a aplicação precisaria rodar em um servidor para testes, além de uma ferramenta específica para testes automatizados de tela.

Existem diversas ferramentas específicas para rodar testes automatizados que envolvam interface do sistema. Neste projeto, a ferramenta escolhida foi o \textit{Selenium}\cite{selenium2018}, amplamente utilizada nos mais diversos \text{frameworks}. Para o Django, a ferramenta específica usada foi o \textit{Selenium WebDriver}, totalmente integrável com as ferramentas de teste. Em contrapartida, soluções desse aspecto necessitam que a aplicação esteje rodando em um servidor, o que não é o caso nos demais testes.

Para resolver o problema de ter um servidor dedicado para testes,o Django oferece uma solução específica chamada \textit{StaticLiveServerTestCase}\cite{staticfiles2018}: uma classe do Django que sobe um servidor exclusivo para os testes escritos na classe que herdá-la. Esta solução atende a necessidade do \textit{Selenium} de ter um servidor dedicado para rodar os testes automatizados.




