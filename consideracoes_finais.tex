\chapter{Considerações Finais}\label{chap:consideracoes-finais}
O projeto foi parcialmente implementado, cobrindo parte do escopo do problema, usando técnicas de Engenharia de Software: modelagem de processos, levantamento de requisitos, casos de uso e verificação/validação. Usou tecnologias de referência no mercado para solucionar o problema e gerou valor ao automatizar etapas do processo existente, permitindo seu crescimento em projetos futuros.

Espera-se também que o projeto entre em atividade para o departamento de Engenharia de Computação da Escola Politécnica da USP, assim como foi o projeto do Portal de Estágios do PCS. O sistema foi projetado para facilitar um processo fundamental para o curso e permitir seu crescimento, em especial as parcerias com empresas.

\section{Resultados Alcançados}
Nesta seção, serão abordados os resultados alcançados pelo sistema, as funcionalidades desenvolvidas e as expectativas alcançadas.

\subsection{Casos de Uso Construídos}
No projeto, doze casos de uso especificados do sistema foram elaborados, construindo toda a parte de gerenciamento de pessoas, disciplinas, atividades, eventos e avaliações, satisfazendo as expectativas dos \textit{stakeholders} e atendendo parte dos problemas encontrados no processo de levantamento de requisitos. Apesar das partes faltantes, o sistema é considerado usável por parte dos coordenadores a partir de 2019.

\section{Próximos Passos}
Alguns casos de uso ficaram de fora das iterações do sistema, sendo eles:

\begin{itemize}
    \item Especificados:
    \begin{itemize}
        \item Listar entregas
        \item Listar necessidades adicionais
    \end{itemize}
    
    \item Não especificados:
    \begin{itemize}
        \item Importar projetos aprovados
        \item Cadastrar projetos avulsos
        \item Buscar projetos anteriores
        \item Propor tema de trabalhos
        \item Buscar temas propostos
    \end{itemize}
\end{itemize}

Além dos casos de uso faltantes, para o sistema ser usado pela comunidade politécnica, resta colocar o sistema na infraestrutura da universidade, tarefa não realizada neste projeto.