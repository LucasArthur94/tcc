\chapter{Considerações Finais}\label{chap:consideracoes-finais}
Neste capítulo, será discutido e listado os resultados do projeto, além das perspectivas de continuidade em projetos futuros.

\section{Perspectivas}
O projeto possui grandes perspectivas, tanto de continuidade, como de uso pela comunidade politécnica. Esta seção pretende discutir um pouco mais sobre o futuro do sistema.

\subsection{Casos de Uso Excedentes}
Alguns casos de uso ficaram de fora das iterações do sistema, sendo eles:

\begin{itemize}
    \item Especificados:
    \begin{itemize}
        \item Listar entregas
        \item Listar necessidades adicionais
    \end{itemize}
    
    \item Não especificados:
    \begin{itemize}
        \item Importar projetos aprovados
        \item Cadastrar projetos avulsos
        \item Buscar projetos anteriores
        \item Propor tema de trabalhos
        \item Buscar temas propostos
    \end{itemize}
\end{itemize}

Além dos casos de uso faltantes, para o sistema ser usado pela comunidade politécnica, resta colocar o sistema na infraestrutura da universidade, tarefa não realizada neste projeto.

\subsection{Legado para a Escola}
A perspectiva maior do projeto está em ser um legado para o departamento de Engenharia de Computação da Escola Politécnica da USP, assim como foi o projeto do Portal de Estágios do PCS. O sistema foi projetado para facilitar um processo fundamental para o curso e permitindo seu crescimento, em especial as parcerias com empresas.

\section{Resultados Alcançados}
Nesta seção final, serão abordados os resultados alcançados pelo sistema, as funcionalidades desenvolvidas e as expectativas alcançadas.

\subsection{Casos de Uso Construídos}
No projeto, diversos casos de uso especificados do sistema foram elaborados, sendo eles:

\begin{itemize}
    \item Login
    \item Cadastrar professores, pessoas externas e alunos
    \item Cadastrar grupos de trabalhos
    \item Cadastrar e editar disciplinas
    \item Entregar atividade
    \item Construir bancas
    \item Avaliar projetos
\end{itemize}

Além disso, foram implementadas as modificações listadas nas reuniões de validação, atendendo os requisitos levantados e os casos de uso descritos.

\subsection{Resultados de Negócio}
Como resultado final, o sistema satisfez as expectativas dos \textit{stakeholders}, atendendo boa parte dos problemas encontrados no processo de levantamento de requisitos. Apesar das partes faltantes, o sistema é considerado usável por parte dos coordenadores já a partir de 2019, o que é um grande feito para o projeto.

