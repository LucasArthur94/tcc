\chapter{Metodologia de Trabalho}\label{chap:metodologia-trabalho}
Neste capítulo, serão discutidos os processos de levantamento de requisitos, a divisão do desenvolvimento em iterações e as reuniões de validação com os principais \textit{stakeholders}.

\section{Processos de Levantamento de Requisitos}
Para o levantamento de requisitos, foram realizadas entrevistas com os \textit{stakeholders} para entender melhor qual é o processo atual, as necessidades encontradas e como o sistema irá ajudar nelas.

\subsection{Entrevistas Iniciais}
A primeira etapa do processo foi a realização de entrevistas com os principais \textit{stakeholders}, de maneira a modelar o processo atual, entender melhor como o sistema irá atuar e estabelecer seus requisitos.

Seguindo os conceitos do capítulo \ref{chap:aspectos-conceituais}, ocorreram entrevistas com os \textit{stakeholders} como forma de levantar requisitos. Por meio dessas entrevistas, o processo geral de negócio (\textit{AS IS}) foi modelado usando BPMN.

Após a modelagem, teve início a construção do Documento Visão, usando como base o processo recém modelado. Com a finalização do Documento Visão, resta entender qual ferramenta convém para a representação de requisitos do sistema.

\subsection{Casos de Uso e Histórias de Usuário}
Realizando um comparativo entre casos de uso e histórias de usuário, tomando como base os conceitos expostos no capítulo \ref{chap:aspectos-conceituais}:

\begin{itemize}
    \item Casos de uso tendem a ser maiores, mais burocráticos e detalhados, ao invés de histórias de usuário, que são mais modulares, simples e flexíveis.
    \item Histórias de usuário demandam maior participação dos \textit{stakeholders}, dada sua maior simplicidade, diferentemente dos casos de uso que, por serem mais completos, servem como base contratual sobre como o sistema deve ser construído.
\end{itemize}

Tomando como base essas diferenças, a solução adotada foram os casos de uso, dada a escassez de tempo de muitos dos \textit{stakeholders}, o que dificultaria bastante o desenvolvimento do sistema com o uso de histórias de usuário.

\section{Divisão em Iterações}
Após a elaboração do Documento Visão, os Casos de Uso elaborados foram divididos em iterações, cada uma com duas reuniões atreladas:

\begin{itemize}
    \item Reunião de revisão dos casos de uso: Os \textit{stakeholders} revisam os casos de uso a serem desenvolvidos, evidenciaram detalhes faltantes nas descrições e priorizaram quais casos de uso da iteração são os mais importantes.
    \item Reunião de validação do sistema: Os \textit{stakeholders} revisam o fluxo do sistema como um todo, se o que foi desenvolvido está conforme com os casos de usos e se há sugestões de alterações e quais são suas prioridades.
\end{itemize}

\section{Processos de Desenvolvimento}
Nesta seção, será explicado um pouco mais sobre ambientes de validação e como eles são usados no processo de desenvolvimento do sistema.

Para desenvolvimento de sistemas, são criados diversos ambientes de aplicação, para testar as funcionalidades com os diversos \textit{stakeholders}, de acordo com a complexidade de negócios do projeto.

Em geral, são três ambientes de aplicação para realizar os processos de validação do sistema, permitindo testes isolados dos \textit{stakeholders}, sem afetar o fluxo de desenvolvimento. São eles\cite{tracyragan2017}:

\begin{itemize}
    \item Desenvolvimento: onde geralmente fica a versão mais recente da aplicação, com as novas funcionalidades desenvolvidas e testadas já integradas no ambiente. Essa versão serve como uma prévia do que será entregue para homologação. As mudanças nesse ambiente são mais frequentes, dado que uma nova funcionalidade completa já pode ir para este ambiente.

    \item Homologação: Já neste ambiente, as mudanças são bem menores e servem como ambiente de validação para os \textit{stakeholders}. No caso do sistema de TCCs, ele serviu como homologação com os coordenadores do curso. Se possível, ele deve ser o mais fiel ao ambiente final de produção, assim o comportamento ideal dele em homologação será o mesmo em produção.

    \item Produção: Já este ambiente é onde o sistema irá rodar, de fato. Esse ambiente deve ser o mais estável possível, com alterações apenas homologadas pelos \textit{stakeholders}. Salvo raras exceções, como falhas e problemas encontrados, nada deve ser colocado aqui sem a aprovação no ambiente de homologação.
\end{itemize}