\chapter{Documento de Visão}\label{chap:vision-doc-appendix}

\section{Introdução}
\subsection{Finalidade e Visão Geral do Documento}

O documento tem como finalidade coletar informações e unificar os pontos de vista sobre o sistema de gerenciamento da disciplina de TCC, como as necessidades encontradas e as causas para tais necessidades.

\subsection{Referências}

O site do departamento, bem como o Tidia-AE serviram como referência auxiliar para entender melhor as soluções atuais.

\section{Posicionamento}
\subsection{Descrição do Problema}

\begin{table}[!htb]
    \centering
    \caption{Descrição básica do problema}
    \label{tab:descricao-problema}
    \resizebox{\textwidth}{!}{\begin{tabular}{|l|l|lll}
        \cline{1-2}
        O problema de         & alto tempo e esforço para gerenciar a disciplina de maneira manual    &  &  &  \\ \cline{1-2}
        afeta                 & orientadores, alunos e coordenadores da disciplina            &  &  &  \\ \cline{1-2}
        cujo impacto é        & demora e esforço para notas, dificuldade para integração entre envolvidos, problemas de demanda     &  &  &  \\ \cline{1-2}
        uma boa solução seria & automatizar o processo e integrar os orientadores no processo &  &  &  \\ \cline{1-2}
    \end{tabular}}
\end{table}

\subsection{Sentença de Posição do Produto}

\begin{table}[!htb]
    \centering
    \caption{Sentença básica de posição do produto}
    \label{sentenca-posicao}
    \resizebox{\textwidth}{!}{\begin{tabular}{|l|l|lll}
        \cline{1-2}
        Para            & alunos, orientadores, coordenadores e técnicos                                                                  &  &  &  \\ \cline{1-2}
        Que             & cursam ou estão envolvidos diretamente                                                            &  &  &  \\ \cline{1-2}
        O sistema       & de gerenciamento dos TCC's                                                                                      &  &  &  \\ \cline{1-2}
        Que             & armazena os TCC's anteriores e gerencia o TCC atual, realizando avaliação e controlando entregas                                                             &  &  &  \\ \cline{1-2}
        Ao contrário do & processo semi-manual de gerenciamento da disciplina, do Tidia-AE e do Site institucional em Wordpress                                                             &  &  &  \\ \cline{1-2}
        O sistema       & permite um acompanhamento dos envolvidos, bem como serve de histórico para os trabalhos anteriores &  &  &  \\ \cline{1-2}
    \end{tabular}}
\end{table}


\section{Descrições dos Envolvidos e Usuários}
\subsection{Resumo dos envolvidos}

Há alguns usuários que não estão diretamente envolvidos com o sistema, mas podem ser beneficiados ou interferem no sistema a ser desenvolvido:

\begin{itemize}
    \item Infraestrutura: Irá aplicar restrições tecnológicas e de negócio dado o ambiente desenvolvido.
    \item Secretaria do Departamento: Nem todos os integrantes irão usar o sistema, mas serão beneficiados pela automação de alguns processos internos da disciplina, diminuindo a demanda recorrente sobre a equipe, vide o fato deles gerenciarem as entregas das avaliações durante o processo.
\end{itemize}

\subsection{Resumo dos usuários}

Há diversos usuários que serão diretamente beneficiados com o sistema. São eles:

\begin{itemize}
    \item Coordenador: Responsáveis por administrarem as disciplinas de TCC 1 e 2 para os cursos de Engenharia de Computação Semestrais e Quadrimestrais.
    \item Orientador: Responsável por construír com os alunos a monografia.
    \item Técnico de Operação: Responsável por administrar novas funcionalidades futuras do sistema.
    \item Técnico do Evento: Responsável por cuidar da parte de infraestrutura dos eventos a serem realizados.
    \item Alunos: Os cursantes das disciplinas de TCC 1 e 2.
    \item Avaliador Teórico: Avaliam os projetos realizados, durante a banca de defesa.
    \item Avaliador Prático: Avaliam os projetos realizados, durante a feira de projetos de formatura.
\end{itemize}

\subsection{Representantes dos usuários}

Foram escolhidos representantes dos perfis citados acima, de maneira a facilitar conversas durante a especificação do documento.

\begin{itemize}
    \item João Batista/Paulo Cugnasca: Coordenadores das duas disciplinas.
    \item Edson Gomi: Professor orientador
    \item Selma: Avaliadora Teórica
    \item Nilton: Técnico do Evento
    \item Michelet: Técnico de Operação
    \item Fábio Levy: Responsável pela infraestrutura de hospedagem.
    \item Bruno Albertini: Responsável pela infraestrutura de hospedagem.
    \item Ex-aluno: Processo do lado do aluno cursante da disciplina.
\end{itemize}

\subsection{Ambiente do usuário}

Para o processo em questão, podemos reforçar alguns pontos importantes:

\begin{itemize}
    \item São 2 professores coordenadores, 2 técnicos, cerca de 50 alunos por ano de TCC e cerca de 20 professores orientadores do departamento de PCS.
    \item O ciclo da disciplinas de TCC dura 1 ano, sendo metade do tempo de especificação e metade de implementação.
    \item Atualmente, há a plataforma Tidia-AE existente para administrar as disciplinas. Porém, ela serve apenas como repositório de arquivos, sem a participação direta dos orientadores e sem infraestrutura de automação e comunicação rápida entre os envolvidos.
    \item Além disso, há o site do departamento, onde ficam as monografias mais recentes, também como simples repositório de arquivos.
\end{itemize}

\subsection{Principais necessidades do usuário}

Diversas necessidades foram encontradas, sendo agrupadas, estudadas e propostas soluções para atendê-las. A lista das necessidades encontradas está a seguir:

\begin{table}[!htb]
  \centering
  \caption{Necessidades encontradas}
  \label{necessidades-encontradas}
  \resizebox{\textwidth}{!}{\begin{tabular}{|l|l|l|l|}
    \hline
    Necessidade                                                                         & Prior. & Sol.Atual         & Sol. Proposta                                                \\ \hline
    Orientador, coordenadores e alunos têm dificuldades em manter contato              & 1      & E-mail, reuniões  & Colocar os orientadores nas entregas                         \\ \hline
    Busca de monografias antigas é complexa.                                            & 2      & Site do PCS       & Criar repositório de monografias finalizadas                 \\ \hline
    O processo de gerenciar empresas participantes é extremamente manual.               & 7      & Conversas         & Integrar empresas às monografias a serem avaliadas           \\ \hline
    Avaliadores têm dificuldade em acessar monografias.                                 & 4      & E-mail            & Permitir monografia de fácil acesso no sistema               \\ \hline
    Avaliação dos projetos é de maneira manual.                                         & 5      & Papel             & Permitir avaliação pelo sistema, inclusive de maneira mobile \\ \hline
    Encontrar temas e alunos e propor temas é um processo manual.                       & 9      & E-mail, conversas & Permitir alunos e orientadores proporem temas no sistema     \\ \hline
    O gerenciamento de recursos pela parte do técnico é depende dos alunos solicitarem. & 8      & E-mail, pen drive & Automatizar as demandas para os técnicos providenciarem      \\ \hline
    O processo de montagem da banca de TCC é manual.                                    & 6      & E-mail, conversas & Permitir montagem de bancas pelo sistema                     \\ \hline
    Fechar as notas finais é um processo exaustivo e com curto prazo.                   & 3      & Papel             & Automatizar o cálculo das notas finais                       \\ \hline
  \end{tabular}}
\end{table}

\subsection{Alternativas e concorrência}
\begin{itemize}
    \item Tidia-AE: Entregas pontuais, com acesso apenas aos integrantes do grupo e aos coordenadores da disciplina. Os orientadores ficam de fora das entregas principais, cabendo aos alunos realizarem essa comunicação de maneira extra-oficial.
    \item Moodle: Plataforma aberta conhecida no mercado para gerenciamento de disciplinas em geral. Serve para construir grupos e até incluir orientadores, porém não é uma alternativa simples e não possui funcionalidades de busca avançada.
    \item Site PCS (Wordpress): Apenas resultado final, de maneira pública, sem integração dos envolvidos no projeto.
\end{itemize}

\section{Visão Geral do Produto}
\subsection{Perspectiva do Produto}
O produto tem como missão automatizar o processo já existente para a disciplina, pois muitas funcionalidades propostas são realizadas de maneira manual ou simplificada, o que demanda muito tempo e esforço dos coordenadores e orientadores, tornando o processo frágil. Além disso, depende da iniciativa de todos os envolvidos, pois todo o processo de comunicação orientador-aluno é feito de maneira isolada do processo coordenador-aluno e coordenador-orientador.

\subsection{Suposições e Dependências}
A principal dependência encontrada é que o serviço deve ser hospedado na infraestrutura interna da Universidade de São Paulo, o que afetará a maneira como o produto será desenvolvido.
  
\section{Recursos do Produto}
O produto a ser desenvolvido deve atender as necessidades descritas anteriormente, ou seja, possíveis conteúdos interessantes são:

\begin{itemize}
    \item Facilitar comunicação orientador/coordenadores, permitindo que eles vejam todas as entregas dos alunos e validem, quando necessário.
    \item Buscar, de maneira pública, as monografias antigas, usando filtros e busca textual.
    \item Incluir avaliadores da feira no sistema para acompanhar monografias correntes.
    \item Permitir avaliação tanto da banca como da feira (com as empresas), permitindo acesso prévio ao conteúdo e facilitando a avaliação (de preferência na plataforma mobile).
    \item Permitir que orientadores, alunos e empresas proponham temas e consigam combinar grupos para realizar as propostas.
    \item Permitir aos técnicos gerenciarem recursos necessários para as apresentações (imprimir apresentações sem necessidade do aluno entregar arquivos via pen drive, por exemplo)
    \item Permitir a montagem de bancas de TCC, convidando os envolvidos e facilitando o acesso ao resultado final do aluno.
\end{itemize}
  
\section{Outros requisitos do produto}
Como principais requisitos não-funcionais importantes, vale ressaltar:

\begin{itemize}
    \item Confiabilidade: O sistema deve permanecer funcional durante as avaliações finais do curso, pois são cruciais para o bom andamento da matéria.
    \begin{itemize}
        \item 
    \end{itemize}
    \item Segurança: Os acessos às monografias em andamento devem ser apenas aos envolvidos diretos do trabalho. Além disso, as empresas só podem acessar o resultado final não revisado, para fins de avaliação.
    \item Usabilidade: O sistema deve ser bem intuitivo e de aprendizagem fácil, pelo tempo curto dos envolvidos na feira e pela mobilidade envolvida (avaliações pelo celular, por exemplo).
    \item Usabilidade: O sistema deve ser bem intuitivo e de aprendizagem fácil, pelo tempo curto dos envolvidos na feira e pela mobilidade envolvida (avaliações pelo celular, por exemplo).
\end{itemize}
