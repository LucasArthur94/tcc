\chapter{Diagramas BPMN}\label{chap:bpmn-appendix}

\section{Introdução}

Os diagramas de Modelo e Notação de Processos de Negócio (Business Process Model and Notation - BPMN) servem para modelar um processo de negócio de maneira a unificar a visão sobre aquele processo e encontrar possíveis otimizações para o mesmo processo.

Para o trabalho em questão, foram usados os diagramas para levantar o processo antes do sistema e após o sistema, assim evidenciamos a presença dele e em que ele ajuda no processo da disciplina.

Para este projeto, foi usada a ferramenta Draw.io\cite{drawio}, para elaborar os diagramas BPMN dos processos citados. Apesar de não ser uma ferramenta específica para esse tipo de diagrama, é uma ferramenta de diagramação simples e que possui bibliotecas de diversos tipos de diagramas, inclusive os de BPMN.

\section{Processo antes do Sistema}

Os diagramas BPMN foram gerados com base nas entrevistas realizadas durante o processo de levantamento de requisitos. Os resultados estão a seguir:

\subsection{TCC 1}
\begin{figure}[H]
    \centering
    \includegraphics[angle=90, origin=c, scale=0.35]{bpmn-tcc1.png}
    \caption{Diagrama BPMN para a disciplina de TCC 1}
    \label{fig:bpmn-tcc1}
\end{figure}

\subsection{TCC 2}
\begin{figure}[H]
    \centering
    \includegraphics[angle=90, origin=c, scale=0.35]{bpmn-tcc2.png}
    \caption{Diagrama BPMN para a disciplina de TCC 2, antes dos eventos finais}
    \label{fig:bpmn-tcc2}
\end{figure}

\subsection{Banca e Feira}
\begin{figure}[H]
    \centering
    \includegraphics[angle=90, origin=c, scale=0.3]{bpmn-banca-feira.png}
    \caption{Diagrama BPMN para os eventos de banca e feira}
    \label{fig:bpmn-banca-feira}
\end{figure}

\subsection{Recuperação}
\begin{figure}[H]
    \centering
    \includegraphics[angle=90, origin=c, scale=0.28]{bpmn-rec.png}
    \caption{Diagrama BPMN para a recuperação da disciplina de TCC 2}
    \label{fig:bpmn-rec}
\end{figure}

\section{Processo com o Sistema}

Já esses diagramas BPMN foram gerados com base no documento de visão, elaborado após o processo de levantamento de requisitos. Os resultados estão a seguir:

TO-DO

