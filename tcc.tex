\documentclass[]{politex}
% ========== Opções ==========
% pnumromarab - Numeração de páginas usando algarismos romanos na parte pré-textual e arábicos na parte textual
% abnttoc - Forçar paginação no sumário conforme ABNT (inclui "p." na frente das páginas)
% normalnum - Numeração contínua de figuras e tabelas
%	(caso contrário, a numeração é reiniciada a cada capítulo)
% draftprint - Ajusta as margens para impressão de rascunhos
%	(reduz a margem interna)
% twosideprint - Ajusta as margens para impressão frente e verso
% capsec - Forçar letras maiúsculas no título das seções
% espacosimples - Documento usando espaçamento simples
% espacoduplo - Documento usando espaçamento duplo
%	(o padrão é usar espaçamento 1.5)
% times - Tenta usar a fonte Times New Roman para o corpo do texto
% noindentfirst - Não indenta o primeiro parágrafo dos capítulos/seções

% ========== Packages ==========
\usepackage[utf8]{inputenc}
\usepackage{amsmath,amsthm,amsfonts,amssymb}
\usepackage{float}
\usepackage{graphicx,hyperref,cite,enumerate,fontawesome,pdfpages, rotating}

% ========== Language options ==========
\usepackage[brazil]{babel}
%\usepackage[english]{babel}

% ========== ABNT (requer ABNTeX 2) ==========
%	http://www.ctan.org/tex-archive/macros/latex/contrib/abntex2
\usepackage[alf]{abntex2cite}

% Forçar o abntex2 a usar [ ] nas referências ao invés de ( )
%\citebrackets{[}{]}

% ========== Lorem ipsum ==========
\usepackage{blindtext}

% ========== PATH Relativo das Imagens ==========
\graphicspath{ {imagens/} }

% ========== Opções do documento ==========
% Título
\titulo{Sistema de Gerenciamento de Disciplinas de Projeto de Formatura do PCS}

% Autor
\autor{Lucas Arthur Felgueiras}

% Para múltiplos autores (TCC)
%\autor{Nome Sobrenome\\%
%		Nome Sobrenome\\%
%		Nome Sobrenome}

% Orientador / Coorientador
\orientador{Prof. Dr. Fábio Levy Siqueira}
% \coorientador{Nome do coorientador (opcional)}

% Tipo de documento
\tcc{de Computação}
%\dissertacao{Engenharia Elétrica}
%\teseDOC{Engenharia Elétrica}
%\teseLD
%\memorialLD

% Departamento e área de concentração
\departamento{Engenharia de Computação e Sistemas Digitais}
\areaConcentracao{Engenharia de Software}

% Local
\local{São Paulo}

% Ano
\data{2018}

\begin{document}
% ========== Capa e folhas de rosto ==========
\capa
\falsafolhaderosto
\folhaderosto


% ========== Folha de assinaturas (opcional) ==========
%\begin{folhadeaprovacao}
%	\assinatura{Prof.\ X}
%	\assinatura{Prof.\ Y}
%	\assinatura{Prof.\ Z}
%\end{folhadeaprovacao}


% ========== Ficha catalográfica ==========
\includepdf[pages=-,pagecommand={},width=\textwidth]{ficha.pdf}


% ========== Dedicatória (opcional) ==========
\dedicatoria{À minha família,
\\
que acreditou em mim
\\
quando nem eu acreditava.}


% ========== Agradecimentos ==========
\begin{agradecimentos}
Agradeço à Deus por toda a força necessária ao longo desses 5 anos de curso para seguir com minha escolha. Agradeço também aos meus familiares: minha irmã Maria Eduarda, minha mãe Márcia, meu pai Francisco, meus avós Helena e Júlio e o cachorro da familia Napoleão, por todo o apoio e dedicação que tiveram comigo durante toda minha jornada.

Agradeço imensamente o Prof. Dr. Fábio Levy Siqueira, pela gigantesca ajuda, paciência e dedicação com a orientação deste trabalho. Agradeço também aos Prof. Dr. João Batista Camargo Júnior e Prof. Dr. Paulo Sérgio Cugnasca pelo tempo reservado nas diversas reuniões que constituíram este projeto.

Agradeço também a minha família de consideração, formada pelos meus amigos de cursinho: César Augusto, Rafael Salazar, Bruno Alves, Diego Rodrigues, Airton Júnior e Júnior Castro; e pelos meus amigos de faculdade e estágios: Victor França, João Vitor, Daniel Norio, Luiz Gustavo, Willian Wener, Martin Andrade, Alexandre Aguiar e Guilherme Eiras.

Por fim, agradeço a todo o departamento de Engenharia de Computação e Sistemas Digitais, em especial aos técnicos e equipe administrativa, pelo empenho e dedicação na formação de grandes profissionais.
\end{agradecimentos}


% ========== Epígrafe (opcional) ==========
\epigrafe{%
	\emph{"Faça ou não faça. Tentativa não há."}
	\begin{flushright}
		-{Mestre Yoda}
	\end{flushright}
}


% ========== Resumo ==========
\begin{resumo}
Ao longo dos anos, o processo das disciplinas de projeto de formatura do PCS ocorre manualmente, com entregas descentralizadas, avaliações em papel, cálculo de notas trabalhoso, entre outras características que compõem um problema real no curso.

O projeto consiste no uso de técnicas de Engenharia de Software para modelar o processo, levantar as necessidades que compõem o problema e construir um sistema responsável por automatizar o processo levantado, com foco na verificação e validação do produto desenvolvido.

A implementação do sistema usa tecnologias renomadas no mercado e cobre parte dos requisitos levantados, gerando valor para os coordenadores das disciplinas e viabilizando seu uso pelo departamento a partir de 2019, com expectativas de se tornar um legado para a Escola Politécnica.
%
\\[3\baselineskip]
%
\textbf{Palavras-Chave} -- modelagem de processos, casos de uso, requisitos, testes, verificação, validação, visão, sistemas web.
\end{resumo}


% ========== Abstract ==========
\begin{abstract}
Over the years, the process of the graduation project disciplines of the PCS occurs manually, with decentralized deliveries, paper assessments, calculation of labor notes, among other characteristics that make up a real problem in the course.

The project consists in the use of software engineering techniques to model the process, to raise the needs that make up the problem and to build a system responsible for automating the raised process, focusing on the verification and validation of the developed product.

The implementation of the system uses renowned technologies in the market and covers part of the requirements raised, generating value for the coordinators of the disciplines and enabling its use by the department from 2019, with expectations of becoming a legacy for the Polytechnic School.
%
\\[3\baselineskip]
%
\textbf{Keywords} -- process modeling, use cases, requirements, tests, verification, validation, vision, web systems.
\end{abstract}


% ========== Listas (opcional) ==========
\listadefiguras
\listadetabelas

% ========== Listas definidas pelo usuário (opcional) ==========

% ========== Sumário ==========
\sumario



% ========== Elementos textuais ==========

\chapter{Introdução}\label{chap:introducao}
Este capítulo é responsável por explicar conceitos iniciais do projeto, as motivações, o objetivo e como este trabalho está organizado.

\section{Motivação}
A motivação deste trabalho está em automatizar o processo relacionado às disciplinas de projeto de formatura do Departamento de Engenharia de Computação e Sistemas Digitais (PCS), que atualmente é manual, trabalhoso e descentralizado.

As disciplinas de projeto de formatura usam de diversos sistemas como base para seu processo, como por exemplo o Tidia-AE. Além disso, toda a parte de avaliação e cálculo das notas é realizado de maneira manual, tanto pelos avaliadores, como pelos coordenadores, o que gera trabalho excessivo de todos os envolvidos.

\section{Objetivo}
O objetivo deste trabalho está em construir um sistema responsável por controlar as disciplinas do projeto de formatura do Departamento de Engenharia de Computação e Sistemas Digitais (PCS), automatizando boa parte do processo manual, bem como centralizar em um único sistema todas as etapas deste processo, evitando o uso de diversos sistemas.

\section{Justificativa}
A maior justificativa do trabalho está no uso amplo de recursos manuais para atender os requisitos do processo das disciplinas de formatura. Toda a parte de atividades, entregas, eventos e avaliações são gerenciadas de maneira manual, com o auxilio de alguns sistemas para cuidar da parte de arquivos submetidos.

\subsection{Sistemas Concorrentes}
Dois sistemas foram analisados como soluções para a motivação proposta: Tidia-AE \href{https://ae4.tidia-ae.usp.br}{https://ae4.tidia-ae.usp.br} e Moodle \href{https://moodle.org/}{https://moodle.org/}.

O Tidia-AE é usado atualmente pelos coordenadores como sistema de suporte para as entregas intermediárias das disciplinas. Para este aspecto, ele atende bem as necessidades, organizando por grupos formados os respoitórios. Porém, não possui uma solução específica para as avaliações, bem como não existe uma maneira de realizar a alocação para os eventos teóricos e práticos

Já o Moodle é uma solução mais robusta, customizável de acordo com as necessidades e com a USP já possuíndo uma solução nesses moldes: o E-Disciplinas \href{https://edisciplinas.usp.br}{https://edisciplinas.usp.br}. A solução é mais poderosa, permitindo avaliações no sistema, bem como a funcionalidade de repositório. Porém, exigiria modificações custosas para permitir as avaliações nos moldes da disciplina, além de outras modificações pesadas que não justificam o uso do Moodle.

Dadas as análises de outras soluções, visando as necessidades observadas acima, há uma justificativa plausível para a construção de um sistema customizado, que atenda às necessidades específicas do processo.

\section{Organização do Trabalho}
Este trabalho está organizado nos seguintes moldes:

\begin{itemize}
    \item Capítulo \ref{chap:aspectos-conceituais}: Trata os aspectos conceituais de Engenharia de Software usados no desenvolvimento deste trabalho.
    \item Capítulo \ref{chap:metodologia-trabalho}: Discute um pouco sobre a metodologia de trabalho usada para a construção do sistema.
    \item Capítulo \ref{chap:especificacao-requisitos-sistema}: Lista os requisitos gerais do sistema.
    \item Capítulo \ref{chap:projeto-implementacao}: Discorre sobre a parte de projeto, arquitetura e implementação do trabalho.
    \item Capítulo \ref{chap:teste-validacao}: Discute um pouco sobre os testes executados e o processo de validação com os coordenadores.
    \item Capítulo \ref{chap:consideracoes-finais}: Mostra as expectativas do trabalho, próximos passos e resultados obtidos.
\end{itemize}


\chapter{Aspectos Conceituais}
\section{Levantamento de Requisitos}
Em Engenharia de Software, é essencial o momento do levantamento de requisitos, dado que são eles que ditam o funcionamento do software a ser desenvolvido, alinha as expectativas dos \textit{stakeholders} e determinam os requisitos funcionais e não funcionais necessários para a aceitação. Neste projeto, para o levantamento de requisitos, foi usada uma das técnicas descritas em \cite{kurtbittnerianspence2002}, que consiste em levantar primeiramente os \textit{stakeholders} do projeto. Segundo \cite{kurtbittnerianspence2002}, a definição traduzida de \textit{stakeholder} é a seguinte:

\begin{citacaoLonga}
"Um indivíduo que é materialmente afetado pelo resultado do
sistema ou o(s) projeto(s) que produzem o sistema."
\end{citacaoLonga}

Ou seja, \textit{stakeholders} não são apenas os indivíduos que efetivamente usarão o sistema, mas sim todos os impactados sua existência. O livro divide os \textit{stakeholders} em 5 grupos:

\begin{itemize}
    \item Usuários: as pessoas que efetivamente usarão o sistema.
    \item Patrocinadores: os financiadores do projeto de software que gerará o sistema.
    \item Desenvolvedores: os responsáveis por desenvolver o sistema levantado pelo processo.
    \item Autoridades: órgãos reguladores que determinam regras para o uso de determinado software.
    \item Consumidores: empresas que compram esses softwares para serem usados.
\end{itemize}

\subsection{Entendimento dos Problemas}

Uma vez mapeado os \textit{stakeholders}, partimos para entender agora as dores que cada um possui e o que eles esperam com o produto final a ser desenvolvido. Alguns processos são sugeridos pelo livro\cite{kurtbittnerianspence2002}, como por exemplo:

\begin{itemize}
    \item Entrevistas: entrevistar os envolvidos e entender diretamente quais são suas dores e expectativas.
    \item Questionários: São úteis quando há um amplo número de \textit{stakeholders} envolvidos.
    \item Grupo Focal: Reunião com alguns representantes dos grupos de \textit{stakeholders} para entender e construir uma visão única sobre o projeto.
    \item Quadros de Aviso: É um tipo particular de grupo focal, onde o quadro serve como unificador da visão, com a diferença de não ter todos reunidos ao mesmo tempo.
    \item \textit{Workshops}: Eventos avisados com antecedência para entender melhor sobre o sistema, com a participação dos envolvidos.
    \item Revisões: Reuniões informais com o intuito de revisar documentos gerados com alguns envolvidos.
    \item Encenação: É uma técnica facilitadora usada em conjunto com \textit{workshops} para obter informações mais específicas ou \textit{feedbacks}.
\end{itemize}

Com essa listagem de problemas levantados pelos processos de levantamento de requisitos, finalmente podemos partir para a visão unificada do processo como um todo e como o software vai atuar no processo. Para isso, é saudável a elaboração de um documento unificando os pontos de vista dos \textit{stakeholders} e estabelecendo o que de fato será o sistema a ser desenvolvido, resultando no Documento de Visão.

\subsection{Documento de Visão}

O documento de visão, segundo o livro\cite{kurtbittnerianspence2002}, traz a seguinte definição (traduzida):

\begin{citacaoLonga}
O Documento de Visão é o artefato do Rational Unified Process\cite{ibm2011} que capta todas as informações de requisitos. Como toda documentação de requisitos, seu objetivo principal é a comunicação.
\end{citacaoLonga}

Existem diversos modelos de Documento de Visão, porém, em sua essência, atendem os seguintes tópicos\cite{kurtbittnerianspence2002}:

\begin{enumerate}
    \item Posicionamento: Como o sistema irá se posicionar no quesito de negócios? Há concorrentes que já resolvem o problema? Quais são seus diferenciais em relação a eles?
    \item \textit{Stakeholders} e usuários: Quem são os envolvidos direta e indiretamente com o desenvolvimento e a existência do sistema?
    \item Necessidades chave: Quais são as demandas que realmente precisam estar nos planos do sistema para satisfazer os envolvidos?
    \item Visão geral do produto: O que é o produto de fato? Quais são suas dependências, capacidades e alternativas ao seu desenvolvimento?
    \item Funcionalidades: Quais são as funcionalidades em alto nível do sistema, para que elas resolvam as necessidades chave listadas anteriormente?
    \item Outros requisitos do produto: Quais são os outros requisitos do sistema que não foram capturados como funcionalidades?
\end{enumerate}

\section{Histórias de Usuário}

Uma das abordagens possíveis para se estabelecer os requisitos levantados e começar a desenvolver de fato o sistema é o uso de Histórias de Usuários (ou \textit{User Stories}). A definição de Histórias de Usuário, traduzido de \cite{jonathanrasmusson}, está a seguir:

\begin{citacaoLonga}
São descrições curtas das funcionalidades que o nosso cliente
gostaria de um dia ver em seu software. Eles geralmente são escritos em pequenos cartões de índice (para nos lembrar de não tentar escrever tudo) e estão lá para nos encorajar a ir falar com nossos clientes.
\end{citacaoLonga}

\subsection{Escopo base da história}
O essencial em escrever boas histórias de usuário está em agregar valor para os \textit{stakeholders}. Ou seja, escrever histórias de usuário que tangem assuntos como arquitetura do sistema, linguagem de implementação, padrões de código entre outros não são boas histórias de usuário, pois salvo raríssimas exceções, o cliente não vê valor em histórias desse tipo.

Em contrapartida, histórias sobre o comportamento do sistema que carregam valor de produto nelas são histórias de usuário importantes para o processo de desenvolvimento. Muitas vezes, porém, histórias de usuário forçam um viés altamente técnico; nessas situações, cabe buscar a causa raiz que levou à "solução" escrita na história: histórias de usuário apresentam problemas de negócio, jamais soluções técnicas.

\subsection{Características básicas}

Boas histórias de usuário costumam ser independentes entre si. Tal independência é importantíssima para mudanças no projeto (que ocorrem com frequência), porque histórias isoladas são fáceis de serem alteradas. Outro ponto importante está em elas serem negociáveis, ou seja, elas podem sofrer alterações e inclusive serem removidas se, com o andamento do projeto, ela perder sua prioridade e tais alterações e remoções não afetam o andar geral das outras tarefas.

Por fim, elas precisam de mais duas características fundamentais: precisam ser testáveis (e quantificar esses testes, se possível), tanto no fluxo de negócio como na escrita de testes automatizados, assim podemos garantir com facilidade a qualidade do que está sendo entregue e; precisam ser pequenas e estimáveis, ou seja, o esforço dela pode ser previsto com antecedência (essa por projetos anteriores ou pela experiência dos integrantes da equipe), permitindo assim medidas de performance dos envolvidos, escopo entregue, entre outros.

\subsection{Um padrão para Histórias de Usuário}

Um bom padrão para escrever histórias de usuário está a seguir\cite{jonathanrasmusson}:

\begin{citacaoLonga}
Eu como $<$para quem é a história$>$
\\
Eu quero $<$o que ele quer$>$
\\
por causa $<$por que ele quer$>$
\end{citacaoLonga}

\section{Casos de Uso}

Como outra alternativa para documentar o que será desenvolvido no sistema, há a solução clássica de desenvolvimento de software: casos de uso. Essencialmente, cada caso de uso possui um ou mais atores agindo com o sistema. Segundo as definições traduzidas de \cite{kurtbittnerianspence2002}:

\begin{citacaoLonga}
\textbf{Atores} representam as pessoas ou coisas que interagem de alguma forma com o sistema; por definição, eles estão fora do sistema. Eles têm um nome e uma breve descrição, e eles estão associados com os casos de uso com os quais eles interagem.
\\
\\
\textbf{Casos de Uso} representam as coisas de valor que o sistema executa para os atores. Casos de uso não são funções ou recursos e não podem ser decompostos. Casos de uso têm um nome e uma breve descrição. Eles também tem descrições detalhadas que são essencialmente histórias sobre como o atores usam o sistema para fazer algo que considerem importante, e que o sistema faz para satisfazer essas necessidades.
\end{citacaoLonga}

\subsection{Características básicas}
Casos de uso, como escrito acima, descrevem as interações entre os atores e o sistema projetado, logo, é coerente que eles estejam relacionados à um processo completo, com começo, meio e fim. Uma grande vantagem de uso dessa técnica está em estabelecer uma espécie de contrato entre a equipe de concepção do sistema e os \textit{stakeholders}, o que evita a participação constante dos envolvidos durante o desenvolvimento.

Um fator interessante está na concepção do documento dos casos de uso, pois ela pode ocorrer de duas formas: como um documento resultante do documento de visão, gerado durante o levantamento de requisitos, ou como uma estratégia para levantar os requisitos e, como consequência, gerar o documento de visão\cite{elisayuminakagawa2013}.

\subsection{Uma estrutura de caso de uso}
Essencialmente, um caso de uso deve possuir a seguinte estrutura\cite{funpar2001}, baseada em \cite{ibm2011}:

\begin{enumerate}
    \item Nome: Nome do Caso de Uso
    \begin{enumerate}
        \item Breve Descrição: Finalidade do caso de uso
    \end{enumerate}
    \item Fluxo de Eventos: A descrição dos eventos que ocorrem no sistema, que são divididos em dois conjuntos principais.
    \begin{enumerate}
        \item Fluxo Básico: É o comportamento básico da interação entre os atores e o sistema. Aqui não devemos ter situações de contorno nem exceções, isso fica ao encargo dos fluxos alternativos.
        \item Fluxos Alternativos: Comportamentos diferenciados do caso de uso, podendo haver mais de um.
    \end{enumerate}
    \item Requisitos Especiais: Requisito não funcional que é específico de um caso de uso, mas que não está contemplado no fluxo de eventos.
    \item Pré-condição: Estado do sistema antes da realização do caso de uso.
    \item Pós-condição: Possíveis estados do sistema após a execução do caso de uso.
\end{enumerate}

Essa estrutura é interessante, pois atende o que deve ser um caso de uso: deve fugir de uma estrutura pequena, englobando um fluxo bem definido do sistema e que traga valor de negócio para os envolvidos. Além disso, com sua descrição comportamental, ele serve como um contrato estabelecido de desenvolvimento, o que abona a participação integral dos \textit{stakeholders}, como é de se esperar de um caso de uso.

\section{Verificação e Validação}

Durante o desenvolvimento, é saudável garantir a qualidade do que está sendo entregue, tanto na parte técnica, quanto no alinhamento com a parte de negócios do projeto. Para isso, existem os processos de verificação e validação, traduzidos do SWEBOK\cite{ieeecomputersociety2014}:

\begin{citacaoLonga}
O objetivo da verificação e validação é ajudar a equipe de desenvolvimento a garantir qualidade no sistema por todo seu ciclo de vida.
\end{citacaoLonga}

\subsection{Verificação}
Em desenvolvimento de software, é uma boa prática acompanhar se os requisitos funcionais e não funcionais estão sendo atendidos, de preferência com acompanhamento das métricas determinadas nos requisitos. Essa prática é conhecida como verificação, definida de maneira resumida pela seguinte pergunta\cite{eduardofigueiredo2018}:

\begin{citacaoLonga}
Estamos construindo o produto corretamente?
\end{citacaoLonga}

\subsection{Validação}
Além de acompanhar os requisitos propostos, é importante entender se o sistema está de acordo com as expectativas operacionais dos \textit{stakeholders}. Essa análise constante de adequação voltada aos negócios é conhecida como validação, definida pela seguinte questão\cite{eduardofigueiredo2018}:

\begin{citacaoLonga}
Estamos construindo o produto correto?
\end{citacaoLonga}

\subsection{Testes}
Em ambos os casos descritos acima, é necessário ter uma ferramenta que automatize esses processos e permita manter a consistência do que foi entregue quando há novas funcionalidades a serem lançadas. Para isso, existe o conceito de testes, onde o sistema passa por uma bateria de exames e revisões para detectar erros e garantir a qualidade e alinhamento do que foi entregue. Vale lembrar que essa bateria de testes é finita, sendo um subgrupo dos infinitos fluxos de execução do domínio\cite{ieeecomputersociety2014}.

Na escala macroscópica, temos os testes de aceitação junto aos \textit{stakeholders}, para validar a conformidade dos requisitos e a qualidade das entregas. Já para testar a arquitetura, costumam ser realizados testes de sistema. Para testar os componentes desenvolvidos do sistema, realizam-se testes de integração entre esses componentes. E, por fim, o código escrito para cada componente elaborado é testado pelos testes de unidade.

\begin{figure}[H]
    \centering
    \includegraphics[scale=0.90]{modelo-v.png}
    \caption{Estrutura de Testes Modelo V\cite{devmedia2013}}
    \label{fig:modelo-v}
\end{figure}

Ao desenvolver um sistema, cada teste deve ser executado a cada nova funcionalidade adicionada e/ou modificada, pois só assim é possível garantir a qualidade do que já foi entregue. Uma boa prática para realizar esses testes, em especial os testes menores, é automatizar a execução deles, o que economiza tempo de uma pessoa dedicada à função de testador.


\chapter{Processo de Desenvolvimento}\label{chap:processo-desenvolvimento}
Neste capítulo, serão discutidos processos de levantamento de requisitos, a divisão do desenvolvimento em iterações, o processo de desenvolvimento, testes e as reuniões de validação com os principais \textit{stakeholders}.

\section{Levantamento de Requisitos}
Para o levantamento de requisitos, foram realizadas entrevistas com os \textit{stakeholders} para entender melhor qual é o processo atual, as necessidades encontradas e como o sistema irá ajudar nelas.

\subsection{Entrevistas Iniciais}
A primeira etapa do processo foi a realização de entrevistas com os principais \textit{stakeholders}, de maneira a modelar o processo atual, entender melhor como o sistema irá atuar e estabelecer seus requisitos.

Seguindo os conceitos do capítulo \ref{chap:aspectos-conceituais}, ocorreram entrevistas com os \textit{stakeholders} como forma de levantar requisitos. Por meio dessas entrevistas, o processo geral de negócio (\textit{AS IS}) foi modelado usando BPMN.

Após a modelagem, teve início a construção do Documento Visão, usando como base o processo recém modelado. Com a finalização do Documento Visão, resta entender como será feita a representação de requisitos do sistema.

\subsection{Casos de Uso e Histórias de Usuário}
Realizando um comparativo entre casos de uso e histórias de usuário, tomando como base os conceitos expostos no capítulo \ref{chap:aspectos-conceituais}:

\begin{itemize}
    \item Casos de uso tendem a ser maiores, mais detalhados, documentando o requisito em detalhe. Já as histórias de usuário são mais modulares e simples, com os detalhes conversados ao longo do projeto, junto com os \textit{stakeholders}.
    \item Casos de uso demandam tempo dos \textit{stakeholders} apenas durante a atividade de levantamento de requisitos (caso não tenha problemas na especificação), já as histórias de usuário, demandam tempo dos \textit{stakeholders} durante a implementação, para se obter mais informações sobre as histórias. 
Casos de uso documentam os requisitos em detalhes, enquanto que não há essa documentação nas histórias do usuário. Os detalhes da história são conversados entre as partes.
\end{itemize}

Tomando como base essas diferenças, a solução adotada foram os casos de uso, dada a escassez de tempo de muitos dos \textit{stakeholders}, o que dificultaria bastante o desenvolvimento do sistema com o uso de histórias de usuário.

Outro fato relevante na decisão para casos de uso está no fato da equipe possuir apenas um único integrante, o que também dificultaria o uso de histórias de usuário. Além disso, este fato implicou na redução do formalismo do processo, dado que muitas etapas requerem mais de um participante nos processos.

\section{Divisão em Iterações}
Após a elaboração do Documento Visão, os Casos de Uso elaborados foram divididos em iterações, cada uma com duas reuniões atreladas:

\begin{itemize}
    \item Reunião de revisão dos casos de uso: os \textit{stakeholders} revisam os casos de uso a serem desenvolvidos, evidenciaram detalhes faltantes nas descrições e priorizaram quais casos de uso da iteração são os mais importantes.
    \item Reunião de validação do sistema: os \textit{stakeholders} revisam o fluxo do sistema como um todo, se o que foi desenvolvido está conforme com os casos de usos e se há sugestões de alterações e quais são suas prioridades.
\end{itemize}

\section{Ambientes de Validação}
Nesta seção, será explicado um pouco mais sobre ambientes de validação e como eles são usados no processo de desenvolvimento do sistema.

Para desenvolvimento de sistemas, são criados diversos ambientes de aplicação, para testar as funcionalidades com os diversos \textit{stakeholders}, de acordo com a complexidade de negócios do projeto.

Em geral, são três ambientes de aplicação para realizar os processos de validação do sistema, permitindo testes isolados dos \textit{stakeholders}, sem afetar o fluxo de desenvolvimento. São eles \cite{tracyragan2017}:

\begin{itemize}
    \item Desenvolvimento: onde geralmente fica a versão mais recente da aplicação, com as novas funcionalidades desenvolvidas e testadas já integradas no ambiente. Essa versão serve como uma prévia do que será entregue para homologação. As mudanças nesse ambiente são mais frequentes, dado que uma nova funcionalidade completa já pode ir para este ambiente.

    \item Homologação: já neste ambiente, as mudanças são bem menores e servem como ambiente de validação para os \textit{stakeholders}. No caso do sistema de TCCs, ele serviu como homologação com os coordenadores do curso. Se possível, ele deve ser o mais fiel ao ambiente final de produção, assim o comportamento ideal dele em homologação será o mesmo em produção.

    \item Produção: este ambiente é onde o sistema irá rodar, de fato. Ele deve ser o mais estável possível, com alterações apenas homologadas pelos \textit{stakeholders}. Salvo raras exceções, como falhas e problemas encontrados, nada deve ser colocado aqui sem a aprovação no ambiente de homologação.
\end{itemize}

\section{Processo de Desenvolvimento e Testes}
Para o processo de desenvolvimento, com os casos de uso definidos e organizados, não há um processo formal, dado o fato da equipe de desenvolvimento ser constituída de apenas um integrante. Cada caso de uso tem seu fluxo básico desenvolvido, junto com os testes relacionados e, ao final do processo, a funcionalidade é enviada para o ambiente de desenvolvimento na nuvem, para testes mais próximos do ambiente de produção.

\chapter{Especificação de Requisitos do Sistema}
Neste capítulo, será abordado um pouco melhor sobre os requisitos do sistema encontrados, os pontos levantados nas diversas reuniões e os requisitos funcionais e não funcionais encontrados. Vale lembrar que, no apêndice \href{chap:vision-doc-appendix}, há maiores detalhes sobre o resultado geral do processo de levantamento de requisitos.

\section{Pontos Levantados nas Reuniões}
Esta seção aborda os diversos pontos levantados nas reuniões, as funcionalidades gerais encontradas e como elas formaram os casos de uso do sistema.

Como principal foco do sistema, tanto o Prof. Dr. João Batista como o Prof. Dr. Pauo Cugnasca querem automatizar o processo atual e permitir seu crescimento, em especial com a participação das empresas, aproveitando a tendência de virutalização.

Algumas funcionalidades gerais levantadas nas reuniões com os \textit{stakeholders}:

\begin{itemize}
    \item Integrar participação do orientador no processo: Facilitar comunicação orientador/coordenadores e permitir acompanhamento mais frequente do orientador.
    \item Base de monografias antigas públicas: Permitir busca e filtro com base em parâmetros.
    \item Incluir empresas para acompanhar monografias: Permitir acesso à monografia antes para realizar avaliação.
    \item Automatizar processo de avaliação: Permitir avaliação tanto da banca como da feira (com as empresas).
    \item Realizar match de temas e orientadores e ideação de temas: Permitir que orientadores, alunos e empresas proponham temas e
    \item combinar grupos de alunos ou alunos individuais e orientadores para realizar trabalhos.
    \item Controle de recursos: Permitir ao Nilton gerenciar recursos necessários para as apresentações (imprimir apresentações sem necessidade do aluno entregar arquivos via pen drive, por exemplo).
    \item Mobilidade: Permitir que avaliações sejam feitas inclusive mobile.
    \item Construção de bancas: Permitir construção de bancas de TCC, para avaliação.
    \item Convite em massa (mala direta): Mala direta para convidar pessoas.
    \item Confiabilidade: resistir a situações adversas (backup de dados).
    \item Hospedagem: deve ser interna, no servidor do PCS USP.
\end{itemize}

Além das funcionalidades gerais, também foi levantado o processo atual das disciplinas, resumido nos seguintes tópicos:
\begin{itemize}
    \item Pré-disciplinas: Conversa no quarto ano para orientar alunos a pensarem nos temas.
    \item Estrutura geral das duas disciplinas: Reuniões intermediárias de acompanhamento, com entregas (apresentações usadas e documentação preliminar correspondente).
    \item Na disciplina de TCC2: Processo de avaliação com Feira e Banca.
    \begin{itemize}
        \item Formação e divulgação de bancas: escolha dos participantes.
        \item Realização da Feira e da Defesa perante banca.
        \item Correção do TCC, com a validação do orientador.
    \end{itemize}
\end{itemize}

\section{Requisitos Funcionais}
Dadas as funcionalidades gerais encontradas, foram levantados os seguintes requisitos funcionais:
\begin{itemize}
    \item Facilitar comunicação entre orientador, coordenadores e alunos.
    \item Buscar as monografias antigas para consulta pública.
    \item Incluir avaliadores da feira para acompanhar monografias correntes.
    \item Permitir avaliação tanto da banca como da feira, permitindo acesso prévio ao conteúdo e facilitando a avaliação.
    \item Permitir que orientadores, alunos e empresas proponham temas e consigam combinar grupos para realizar as propostas.
    \item Permitir aos técnicos gerenciarem recursos necessários para as apresentações.
    \item Permitir a montagem de bancas de TCC.
\end{itemize}

\section{Requisitos Não Funcionais}
Além dos requisitos funcionais, alguns requisitos não funcionais importantes foram levantados aqui:

\begin{itemize}
    \item Confiabilidade: O sistema deve permanecer funcional durante as avaliações finais do curso, pois são cruciais para o bom andamento da matéria.
    \item Segurança: Os acessos às monografias em andamento devem ser apenas aos envolvidos diretos do trabalho. Além disso, as empresas só podem acessar o resultado final não revisado, para fins de avaliação. Foco especial nos requisitos de Confidencialidade e Integridade.
    \item Confiabilidade: O sistema deve suportar situações de falha e lidar bem com as informações, garantindo sua proteção. No caso, foco especial em Recuperabilidade.
    \item Usabilidade: O sistema deve ser bem intuitivo e de aprendizagem fácil, pelo tempo curto dos envolvidos na feira e pela mobilidade envolvida (avaliações pelo celular, por exemplo). Foco especial na Operacionalidade, Estética/Interface e Proteção contra Erros do Usuário.
    \item Manutenibilidade: O sistema deve ser bem intuitivo de realizar alterações, instalações e afins, dado que alunos podem tocar sua manutenção futuramente. Foco especial na Modificabilidade.
\end{itemize}

\chapter{Projeto e Implementação}
Neste capítulo, entraremos em mais detalhes da parte de projeto e implementação do sistema, com base nos requisitos já levantados, passando pelas tecnologias envolvidas e pela arquitetura do sistema.

\section{Elaboração e Estrutura do Sistema}
Está seção explica, com mais detalhes, os casos de uso do sistema, as responsabilidades em comum entre cada um deles e como isso pode ser estruturado na arquitetura.

\subsection{Casos de Uso}
Com bases nos requisitos levantados anteriormente, foram construídos os casos de uso a seguir (para mais detalhes, ver o apêndice \href{chap:use-case-appendix}):

\begin{itemize}
    \item Propor tema de trabalho
    \item Login
    \item Cadastrar disciplinas
    \item Editar disciplinas
    \item Cadastrar grupos de trabalhos
    \item Cadastrar professores
    \item Cadastrar pessoas externas
    \item Entregar atividade
    \item Listar entregas
    \item Listar necessidades adicionais
    \item Construir bancas práticas
    \item Construir bancas teóricas
    \item Avaliar projetos práticos
    \item Avaliar monografias teóricas
    \item Calcular nota final dos projetos
    \item Importar projetos aprovados
    \item Cadastrar projetos avulsos
    \item Buscar projetos anteriores
    \item Buscar temas propostos
\end{itemize}

\subsection{Diagramas de Casos de Uso}
Os casos de uso foram estruturados nos seguintes diagramas (elaborados com a ajuda da ferramenta PlantText \cite{planttext2018}, para ver como elas foram geradas, ver apêndice \href{chap:use-case-appendix}):

\begin{figure}[H]
    \centering
    \includegraphics[scale=0.6]{atv.png}
    \caption{Diagrama de Casos de Uso para Atividades}
    \label{fig:use-case-atv}
\end{figure}

\begin{figure}[H]
    \centering
    \includegraphics[scale=0.4]{bib.png}
    \caption{Diagrama de Casos de Uso para Base de Dados (Biblioteca Virtual)}
    \label{fig:use-case-atv}
\end{figure}

\begin{figure}[H]
    \centering
    \includegraphics[scale=0.3]{ev-ava.png}
    \caption{Diagrama de Casos de Uso para Eventos e Avaliações}
    \label{fig:use-case-atv}
\end{figure}

\begin{figure}[H]
    \centering
    \includegraphics[scale=0.6]{infra.png}
    \caption{Diagrama de Casos de Uso para Infraestrutura}
    \label{fig:use-case-atv}
\end{figure}

\begin{figure}[H]
    \centering
    \includegraphics[scale=0.5]{recursos.png}
    \caption{Diagrama de Casos de Uso para Gerenciamento dos Recursos}
    \label{fig:use-case-atv}
\end{figure}

\section{Tecnologias Utilizadas}
Após determinar os requisitos do sistema e estabelecer os casos de uso, resta determinar as tecnologias a serem usadas, bem como a arquitetura do sistema.

Como parte dos requisitos, era necessário o uso de sistemas \textit{web} para os \textit{stakeholders} terem maior mobilidade e facilidade ao realizarem as interações com o sistema. Sendo assim, é valido passar por alguns conceitos tecnológicos do projeto.

\subsection{\textit{Frameworks web} e o Django}
É comum, em desenvolvimento de sistemas, usar soluções prontas como forma de simplificar o desenvolvimento de soluções. Essas soluções são conhecidas como arcabouços (ou \textit{frameworks})\cite{iansommerville2011}: "Um framework é uma estrutura genérica estendida para se criar uma aplicação ou subsistema mais específico.". Para este projeto, foi usado um \textit{framework}, em linguagem Python, chamado \textbf{Django}.

O Django é estruturado em aplicações (apps), onde cada aplicação corresponde a uma parte do sistema, geralmente independente e reciclável (ou seja, pode ser usada em aplicações Django diferentes). Cada aplicação segue o padrão \textit{Model-View-Controller}.

O padrão \textit{Model-View-Controller} - MVC é um padrão arquitetural para organizar os componentes da aplicação. Ele é dividido em três grandes grupos\cite{thedjangobook2018}:

\begin{itemize}
    \item Modelo (\textit{Model}): Uma representação (interface) para os dados da aplicação.
    \item Visualização (\textit{View}): Camada de apresentação dos dados da aplicação. No caso do Django, ele é chamado de \textit{Template}.
    \item Controlador (\textit{Controller}): Camada de controle que interliga o modelo com a apresentação dos dados, onde geralmente fica a lógica de negócio. No caso do Django, ele é conhecido como \textit{View}.
\end{itemize}

A vantagem do padrão está no fluxo de dados que existe entre os grupos, além de evitar códigos com funcionalidades diferentes em lugares errados, como, por exemplo, lógica de negócio na camada de apresentação.

\subsection{Arquitetura}
O Django foi usado na versão mais recente estável (2.1), que corrigiu algumas novidades da versão 2.0. É esta versão que foi usada pelo projeto em questão, dado que é estável e possui suporte de apoio da equipe até Dezembro de 2020\cite{djangodownload}. Ele usa, como padrão, o SQLite, o que atendeu bem durante o desenvolvimento. Já o Heroku tem como banco de dados padrão o PostgreSQL, o que exigiu o chaveamento entre os bancos no ambiente local de desenvolvimento e os ambientes de validação hospedados no serviço. Por isso, o projeto possui configurado dois pacotes de gerenciamento de banco de dados, um para SQLite (nativo do Django), outro para PostgreSQL (o Psycopg\cite{lucassouto2017}).

Para controlar os acessos ao sistema, por limitações do Django, cada pessoa terá um único usuário, com vários perfis diferentes de acesso, cada qual com suas permissões e ações. São quatro perfis diferentes de acesso: estudante, docente, convidado e coordenador. Cada usuário pode ter um ou mais perfis de acesso simultâneos (é o caso, por exemplo, dos coordenadores da disciplina, que também são docentes e podem orientar e avaliar projetos).

\subsection{Aplicações Construídas}
O sistema foi estruturado em aplicações menores \textit{apps}, cada uma com uma responsabilidade:

\begin{itemize}
    \item home: Aplicação responsável por gerenciar as funções básicas a todos os usuários, como por exemplo login/logout, o conteúdo da página de entrada, entre outros.
    \item users: Gerencia os usuários cadastrados do sistema e os seus perfis (aluno, docente, convidado e coordenador)
    \item disciplines: Gerencia as disciplinas de TCC do curso, duas disciplinas por ano (TCC1 e TCC2).
    \item activities: Gerencia as atividades das diversas disciplinas do sistema.
    \item workgroups: Cuida dos grupos de trabalho criados durante as disciplinas. 
    \item deliveries: Cuida das entregas das atividades do curso, cada uma realizada pelos grupos de trabalho e revisadas pelos orientadores/co-orientadores.
    \item rooms: Cuida das salas que serão usadas nos eventos de bancas teóricas e práticas.
    \item events: Gerencia os eventos teóricos e práticos que ocorrem nas disciplinas de TCC2.
    \item allocations: Cuida das alocações de cada grupo, para cada evento teórico ou prático, junto com os convidados e docentes que avaliarão o grupo.
    \item evaluations: Gerencia as avaliações que cada docente ou convidado alocado fará nos eventos.
    \item rules: Determina as disciplinas e os eventos que serão usados para calcular as notas finais.
    \item score: Gerencia as notas finais calculadas para cada situação, por grupo de trabalho.
\end{itemize}

\chapter{Teste e Validação}

\section{Validações}
\section{Testes Executados}
\section{Ambientes de Validação}

\chapter{Considerações Finais}\label{chap:consideracoes-finais}
Neste capítulo, será discutido e listado os resultados do projeto, além das perspectivas de continuidade em projetos futuros.

\section{Perspectivas}
O projeto possui grandes perspectivas, tanto de continuidade, como de uso pela comunidade politécnica. Esta seção pretende discutir um pouco mais sobre o futuro do sistema.

\subsection{Casos de Uso Excedentes}
Alguns casos de uso ficaram de fora das iterações do sistema, sendo eles:

\begin{itemize}
    \item Especificados:
    \begin{itemize}
        \item Listar entregas
        \item Listar necessidades adicionais
    \end{itemize}
    
    \item Não especificados:
    \begin{itemize}
        \item Importar projetos aprovados
        \item Cadastrar projetos avulsos
        \item Buscar projetos anteriores
        \item Propor tema de trabalhos
        \item Buscar temas propostos
    \end{itemize}
\end{itemize}

Além dos casos de uso faltantes, para o sistema ser usado pela comunidade politécnica, resta colocar o sistema na infraestrutura da universidade, tarefa não realizada neste projeto.

\subsection{Legado para a Escola}
A perspectiva maior do projeto está em ser um legado para o departamento de Engenharia de Computação da Escola Politécnica da USP, assim como foi o projeto do Portal de Estágios do PCS. O sistema foi projetado para facilitar um processo fundamental para o curso e permitindo seu crescimento, em especial as parcerias com empresas.

\section{Resultados Alcançados}
Nesta seção final, serão abordados os resultados alcançados pelo sistema, as funcionalidades desenvolvidas e as expectativas alcançadas.

\subsection{Casos de Uso Construídos}
No projeto, diversos casos de uso especificados do sistema foram elaborados, sendo eles:

\begin{itemize}
    \item Login
    \item Cadastrar professores, pessoas externas e alunos
    \item Cadastrar grupos de trabalhos
    \item Cadastrar e editar disciplinas
    \item Entregar atividade
    \item Construir bancas
    \item Avaliar projetos
\end{itemize}

Além disso, foram implementadas as modificações listadas nas reuniões de validação, atendendo os requisitos levantados e os casos de uso descritos.

\subsection{Resultados de Negócio}
Como resultado final, o sistema satisfez as expectativas dos \textit{stakeholders}, atendendo boa parte dos problemas encontrados no processo de levantamento de requisitos. Apesar das partes faltantes, o sistema é considerado usável por parte dos coordenadores já a partir de 2019, o que é um grande feito para o projeto.




% ========== Referências ==========
% --- IEEE ---
%	http://www.ctan.org/tex-archive/macros/latex/contrib/IEEEtran
%\bibliographystyle{IEEEbib}

% --- ABNT (requer ABNTeX 2) ---
%	http://www.ctan.org/tex-archive/macros/latex/contrib/abntex2
% \bibliographystyle{abntex2-num}

\bibliography{refs}

% ========== Apêndices (opcional) ==========
\apendice
\chapter{Manual do Usuário}\label{chap:user-manual-appendix}
\chapter{Diagramas BPMN}\label{chap:bpmn-appendix}

\section{Introdução}

Os diagramas de Modelo e Notação de Processos de Negócio (Business Process Model and Notation - BPMN) servem para modelar um processo de negócio de maneira a unificar a visão sobre aquele processo e encontrar possíveis otimizações para o mesmo processo.

Para o trabalho em questão, foram usados os diagramas para levantar o processo antes do sistema, assim evidenciamos os pontos onde o sistema pode agir.

Para este projeto, foi usada a ferramenta Draw.io\cite{drawio}, para elaborar os diagramas BPMN dos processos citados. Apesar de não ser uma ferramenta específica para esse tipo de diagrama, é uma ferramenta de diagramação simples e que possui bibliotecas de diversos tipos de diagramas, inclusive os de BPMN.

\section{Processo antes do Sistema}

Os diagramas BPMN foram gerados com base nas entrevistas realizadas durante o processo de levantamento de requisitos. Vale lembrar que eles foram divididos de maneira a facilitar a visualização neste documento.

\subsection{TCC 1}
\begin{figure}[H]
    \centering
    \includegraphics[angle=90, origin=c, scale=0.50]{bpmn-tcc1-pt1.png}
    \caption{Diagrama BPMN para a disciplina de TCC 1 - Parte 1}
    \label{fig:bpmn-tcc1-pt1}
\end{figure}

\begin{figure}[H]
    \centering
    \includegraphics[angle=90, origin=c, scale=0.70]{bpmn-tcc1-pt2.png}
    \caption{Diagrama BPMN para a disciplina de TCC 1 - Parte 2}
    \label{fig:bpmn-tcc1-pt2}
\end{figure}

\subsection{TCC 2 - Sem eventos finais}
\begin{figure}[H]
    \centering
    \includegraphics[angle=90, origin=c, scale=0.60]{bpmn-tcc2-pt1.png}
    \caption{Diagrama BPMN para a disciplina de TCC 2 - Parte 1}
    \label{fig:bpmn-tcc2-pt1}
\end{figure}

\begin{figure}[H]
    \centering
    \includegraphics[angle=90, origin=c, scale=0.70]{bpmn-tcc2-pt2.png}
    \caption{Diagrama BPMN para a disciplina de TCC 2 - Parte 2}
    \label{fig:bpmn-tcc2-pt2}
\end{figure}

\subsection{Banca e Feira}
\begin{figure}[H]
    \centering
    \includegraphics[angle=90, origin=c, scale=0.55]{bpmn-banca-feira-pt1.png}
    \caption{Diagrama BPMN para a Banca e Feira - Parte 1}
    \label{fig:bpmn-banca-feira-pt1}
\end{figure}

\begin{figure}[H]
    \centering
    \includegraphics[angle=90, origin=c, scale=0.60]{bpmn-banca-feira-pt2.png}
    \caption{Diagrama BPMN para a Banca e Feira - Parte 2}
    \label{fig:bpmn-banca-feira-pt2}
\end{figure}

\subsection{Recuperação}
\begin{figure}[H]
    \centering
    \includegraphics[angle=90, origin=c, scale=0.45]{bpmn-rec-pt1.png}
    \caption{Diagrama BPMN para a Recuperação - Parte 1}
    \label{fig:bpmn-rec-pt1}
\end{figure}

\begin{figure}[H]
    \centering
    \includegraphics[angle=90, origin=c, scale=0.60]{bpmn-rec-pt2.png}
    \caption{Diagrama BPMN para a Recuperação - Parte 2}
    \label{fig:bpmn-rec-pt2}
\end{figure}

\chapter{Casos de Uso}\label{chap:use-case-appendix}
Neste apêndice consta os casos de uso escritos para o sistema em questão, usando o padrão explicado no capítulo de casos de uso \cite{ibm2011}. Para economizar espaços, campos ausentes nos casos de uso não foram explicitados.

\section{Cadastrar disciplinas}
\begin{enumerate}
    \item Breve Descrição: Coordenadores cadastram a disciplina, os alunos participantes e as atividades.
    \item Fluxo Básico:
        \begin{itemize}
            \item Coordenador insere disciplina, com os seguintes dados:
            \begin{itemize}
                \item Modalidade (Sem/Quad)
                \item Data de início e data de término
            \end{itemize}
            \item Coordenador importa planilha com alunos participantes, com os seguintes dados dos alunos
            \begin{itemize}
                \item Nome
                \item E-mail
                \item Nº USP
            \end{itemize}
            \item Coordenador insere uma nova atividade da disciplina, com os seguintes dados
            \begin{itemize}
                \item Nome
                \item Data de entrega e arquivos relacionados
                \item Peso da atividade
            \end{itemize}
            \item Sistema salva alunos novos, dispara e-mail ao aluno com seu acesso (login e senha) e vincula os existentes à disciplina e salva a disciplina
        \end{itemize}
    \item Fluxos Alternativos:
    \begin{enumerate}
        \item Data de início é posterior a de término
        \begin{enumerate}
            \item Sistema exibe novamente tela de cadastro da disciplina, alertando sobre erro
        \end{enumerate}
        \item E-mail é inválido
        \begin{enumerate}
            \item Sistema exibe novamente tela de cadastro da disciplina, alertando sobre erro
        \end{enumerate}
        \item E-mail retornou
        \begin{enumerate}
            \item Sistema envia e-mail ao coordenador com o aluno problemático
        \end{enumerate}
    \end{enumerate}
    \item Pré-condição: Coordenador deve estar logado
\end{enumerate}

\section{Editar disciplinas}
\begin{enumerate}
    \item Breve Descrição: Coordenadores editam disciplinas, cadastrando atividades, editando alunos etc.
    \item Fluxo Básico:
        \begin{itemize}
            \item Coordenador edita início e término da disciplina, alunos participantes, com novos alunos ou removendo os já participantes
            \item Coordenador insere uma nova atividade da disciplina, com os seguintes dados
            \begin{itemize}
                \item Nome
                \item Data de entrega
                \item Arquivos relacionados
                \item Peso da atividade
            \end{itemize}
            \item Sistema salva novas atividades e as mudanças da disciplina
        \end{itemize}
    \item Pré-condição: Coordenador deve estar logado
\end{enumerate}

\section{Cadastrar professores}
\begin{enumerate}
    \item Breve Descrição: Coordenadores cadastram professores do departamento que podem orientar/co-orientar.
    \item Fluxo Básico:
    \begin{itemize}
        \item Coordenador insere os dados do professor
        \begin{itemize}
            \item Nome
            \item Número USP
            \item E-mail
        \end{itemize}
        \item Sistema salva o professor, disparando e-mail para o professor cadastrado
    \end{itemize}
    \item Pré-condição: Coordenador deve estar logado
\end{enumerate}

\section{Cadastrar convidados externos}
\begin{enumerate}
    \item Breve Descrição: Coordenadores cadastram convidados externos do departamento que podem co-orientar.
    \item Fluxo Básico:
    \begin{itemize}
        \item Coordenador insere os dados do convidado externo
        \begin{itemize}
            \item Nome
            \item Empresa
            \item E-mail
        \end{itemize}
        \item Sistema salva o convidado externo, disparando e-mail para o convidado cadastrado
    \end{itemize}
    \item Pré-condição: Coordenador deve estar logado
\end{enumerate}

\section{Cadastrar grupos de trabalhos}
\begin{enumerate}
    \item Breve Descrição: Coordenadores cadastram os grupos com os temas e os orientadores, com a confirmação da participação do orientador no grupo.
    \item Fluxo Básico:
    \begin{itemize}
        \item Coordenador insere os dados do grupo
        \begin{itemize}
            \item Título
            \item Alunos
            \item Orientador
            \item Co-orientador
        \end{itemize}
        \item Sistema salva grupo e envia e-mail para o orientador, co-orientador e alunos
        \item Orientador valida participação no grupo
        \item Co-orientador valida participação no grupo
    \end{itemize}
    \item Fluxos Alternativos:
    \begin{enumerate}
        \item Grupo não tem orientador
        \begin{enumerate}
            \item Sistema cadastra grupo, enviando e-mail para os alunos com o aviso de urgência na escolha do orientador
        \end{enumerate}
        \item Grupo não tem aluno
        \begin{enumerate}
            \item Sistema retorna para a tela de cadastro de grupo, avisando sobre o erro de ausência de alunos
        \end{enumerate}
    \end{enumerate}
    \item Pré-condição: Coordenador deve estar logado, alunos, orientadores e co-orientadores cadastrados
    \item Pós-condição: Grupo confirmado
\end{enumerate}

\section{Login}
\begin{enumerate}
    \item Breve Descrição: Alunos, Orientadores, Co-orientadores e Coordenadores acessam sistema de maneira tradicional ou via Senha Única USP (para pertencentes à USP).
    \item Fluxo Básico:
    \begin{itemize}
        \item Sistema mostra campos de login e senha
        \item Usuário insere seu e-mail e sua senha
        \item Sistema valida e-mail e senha
        \item Usuário acessa sistema
    \end{itemize}
    \item Fluxos Alternativos:
    \begin{enumerate}
        \item Usuário erra credenciais
        \begin{enumerate}
            \item Sistema retorna para tela de acesso ao sistema, exibindo mensagem de erro
            \item Retorna normalmente à situação de mostrar campos de login e senha
        \end{enumerate}
        \item Usuário realiza login pela Senha Única da USP
        \begin{enumerate}
            \item Usuário seleciona $``$Acessar pela Senha Única USP$"$ 
            \item Usuário é redirecionado para os sistemas USP
            \item Retorna para a situação de acesso ao sistema
        \end{enumerate}
    \end{enumerate}
    \item Requisitos Especiais: Integração via Shibboleth com os Sistemas USP
    \item Pós-condição: Usuário dentro do sistema
\end{enumerate}

\section{Entregar atividade}
\begin{enumerate}
    \item Breve Descrição: Alunos submetem no Google Drive arquivos da atividade para a leitura do orientador, co-orientador e coordenadores. Orientador e co-orientador revisam e dão seu aval de aprovação com a documentação.
    \item Fluxo Básico:
    \begin{itemize}
        \item Aluno submete arquivos nos respectivos espaços de atividades, que são carregados no Google Drive e deixa comentários adicionais sobre a entrega
        \item Sistema salva a entrega com o status da entrega da atividade para $``$Não avaliada$"$
        \item Sistema envia e-mail para Orientador e Co-orientador avisando de submissão
        \item Orientador baixa documentos submetidos
        \item Orientador avalia a entrega, faz comentários aos alunos e comentários exclusivos à coordenação com notas
        \item Sistema salva entrega e dispara e-mail com o resultado da avaliação para os alunos
    \end{itemize}
    \item Fluxos Alternativos:
    \begin{enumerate}
        \item Data de submissão expirou (1)
        \begin{enumerate}
            \item Aluno não consegue interagir com atividade, encerrando fluxo
        \end{enumerate}
        \item Co-orientador realiza fluxo de revisão, antes do Orientador (4)
        \begin{enumerate}
            \item Passos 5 - 7 ocorrem normalmente, trocando Orientador por Co-orientador
        \end{enumerate}
        \item Co-orientador realiza fluxo de revisão, após Orientador (8)
        \begin{enumerate}
            \item Sistema exibe detalhes da entrega, porém não permite edições do lado do Co-orientador, encerrando fluxo
        \end{enumerate}
        \item Aluno submete nova entrega da atividade após ter feito uma submissão (1)
        \begin{enumerate}
            \item Aluno vê status da avaliação
            \item Aluno realiza nova entrega da atividade, repetindo o caso de uso
        \end{enumerate}
        \item Aluno submete entrega da atividade quando alguém do grupo já submeteu (1)
        \begin{enumerate}
            \item Sistema exibe detalhes da entrega já realizada
            \item Aluno pode realizar nova entrega da atividade, passando por cima da entrega anterior e repetindo o caso de uso
        \end{enumerate}
    \end{enumerate}
    \item Pré-condição: Atores devem estar autenticados e atividade deve estar cadastrada no sistema
\end{enumerate}


\section{Construir bancas práticas}
\begin{enumerate}
    \item Breve Descrição: Coordenadores selecionam os participantes da banca prática, já cadastrados no sistema, e os notifica com comentários sobre o evento.
    \item Fluxo Básico:
    \begin{itemize}
        \item Coordenador informa o dia da feira, a disciplina correspondente e as salas disponíveis para o evento, além de determinar o peso da avaliação da banca
        \item Sistema salva evento
        \item Coordenador seleciona evento recém criado.
        \item Sistema lista grupos do evento.
        \item Coordenador seleciona o grupo que deseja atribuir a sala e os convidados.
        \item Coordenador seleciona os convidados que avaliarão o grupo
        \item Sistema salva banca e envia e-mail para os convidados externos, avisando-os sobre sua participação
        \item Convidado acessa sistema e confirma participação na banca
    \end{itemize}
    \item Pré-condição: Atores devem estar autenticados e grupo de trabalho deve estar cadastrado no sistema
\end{enumerate}

\section{Construir bancas teóricas}
\begin{enumerate}
    \item Breve Descrição: Coordenadores escolhem participantes da banca teórica do grupo, selecionam o presidente da banca, realizam o agendamento do horário, validando inconsistências (participante já possui horário ocupado) e notificam os participantes por e-mail.
    \item Fluxo Básico:
    \begin{itemize}
        \item Coordenador informa o dia da banca, a disciplina correspondente e as salas disponíveis para o evento
        \item Sistema salva evento
        \item Coordenador seleciona evento recém criado.
        \item Sistema lista grupos do evento.
        \item Coordenador seleciona o grupo que deseja atribuir a sala e os convidados.
        \item Coordenador seleciona os avaliadores da banca e o horário da avaliação. O orientador é um dos pré-selecionados por padrão
    \end{itemize}
    \item Fluxos Alternativos:
    \begin{enumerate}
        \item Convidado externo já possui banca nesse dia e horário
        \begin{enumerate}
            \item Sistema barra criação de banca, alertando sobre qual convidado já possui agenda ocupada
        \end{enumerate}
        \item Sala está ocupada no horário selecionado
        \begin{enumerate}
            \item Sistema barra criação de banca, alertando sobre qual sala já possui agenda ocupada
        \end{enumerate}
    \end{enumerate}
    \item Pré-condição: Atores devem estar autenticados e grupo de trabalho deve estar cadastrado no sistema
\end{enumerate}

\section{Listar entregas}
\begin{enumerate}
    \item Breve Descrição: Técnicos recebem os arquivos de impressão, com normalização do título, separados por grupo.
    \item Fluxo Básico:
    \begin{itemize}
        \item Coordenador lista todas as entregas finais de impressão (\textit{banner} e \textit{press-release})
        \item Sistema salva as listas de arquivos finais, com o nome normalizado e envia por e-mail para o técnico responsável
    \end{itemize}
    \item Fluxos Alternativos:
    \begin{enumerate}
        \item Grupo de trabalho está com arquivo faltante
        \begin{enumerate}
            \item Sistema envia e-mail para o grupo com entrega faltante, avisando para regularizar a situação com urgência.
        \end{enumerate}
    \end{enumerate}
    \item Pré-condição: Atores devem estar autenticados e grupo de trabalho deve estar cadastrado no sistema
\end{enumerate}

\section{Listar necessidades adicionais}
\begin{enumerate}
    \item Breve Descrição: Técnicos recebem necessidades adicionais revisadas pelos orientadores, separadas por grupo.
    \item Fluxo Básico:
    \begin{itemize}
        \item Coordenador lista todos os comentários das entregas finais
        \item Sistema salva a lista de comentários e a envia por e-mail para o técnico responsável
    \end{itemize}
    \item Pré-condição: Atores devem estar autenticados e grupo de trabalho deve estar cadastrado no sistema
\end{enumerate}

\section{Avaliar projetos práticos}
\begin{enumerate}
    \item Breve Descrição: Participantes da banca prática avaliam os projetos que estão envolvidos, limitando avaliações até o final do dia.
    \item Fluxo Básico:
    \begin{itemize}
        \item Convidado externo acessa espaço da banca, com detalhes do grupo e as entregas finais
        \item Convidado preenche os comentários e as notas de acordo com cada critério estabelecido para avaliação de bancas práticas
        \item Convidado salva avaliação
    \end{itemize}
    \item Fluxos Alternativos:
    \begin{enumerate}
        \item Convidado tenta submeter avaliação em dia diferente ao da banca prática
        \begin{enumerate}
            \item Avaliação é barrada, avisando o convidado de que a avaliação só pode ser feita exclusivamente no dia
        \end{enumerate}
    \end{enumerate}
    \item Pré-condição: Atores devem estar autenticados e grupo de trabalho deve estar cadastrado no sistema
\end{enumerate}


\section{Avaliar monografias teóricas}
\begin{enumerate}
    \item Breve Descrição: Participantes da banca teórica avaliam as monografias que estão envolvidas, gerando comentários e definindo o status do trabalho (aprovado, aprovado com correções, recuperação e reprovado), limitando avaliações até o final do dia.
    \item Fluxo Básico:
    \begin{itemize}
        \item Participante da banca acessa espaço da banca, com detalhes do grupo e as entregas parciais e finais.
        \item Participante preenche os comentários e as notas de acordo com cada critério estabelecido para avaliação de bancas teóricas.
        \item Participante salva avaliação, com seu parecer para aprovação da monografia.
        \item Sistema usa a média das avaliações da banca para determinar o estado do grupo.
    \end{itemize}
    \item Pré-condição: Atores devem estar autenticados e grupo de trabalho deve estar cadastrado no sistema
\end{enumerate}


\section{Calcular nota final dos projetos}
\begin{enumerate}
    \item Breve Descrição: Coordenadores determinam a fórmula para calcular as notas finais, com base nas entregas parciais durante as duas disciplinas e o sistema calcula as notas finais de todos os grupos participantes.
    \item Fluxo Básico:
    \begin{itemize}
        \item Coordenador escolhe a disciplina de TCC2, a banca teórica e a feira prática que deseja obter as entregas
        \item Sistema salva fórmula de avaliação
        \item Sistema lista todos os grupos, com as notas calculadas e os estados de avaliação de cada grupo
    \end{itemize}
    \item Fluxos Alternativos:
    \begin{enumerate}
        \item Coordenador não preenche algum dos campos necessários
        \begin{enumerate}
            \item Sistema barra criação de fórmula de avaliação, avisando os campos faltantes
        \end{enumerate}
        \item Grupo tem alguma avaliação faltante
        \begin{enumerate}
            \item Sistema exibe grupo, mas com campo de Não Avaliado
        \end{enumerate}
        \item Grupo vai para recuperação na avaliação da banca teórica
        \begin{enumerate}
            \item A nota calculada é a nota da banca teórica, passando todas as outras avaliações realizadas ao longo das disciplinas
        \end{enumerate}
        \item Grupo é reprovado na avaliação da banca teórica
        \begin{enumerate}
            \item A nota calculada é a nota da banca teórica, passando todas as outras avaliações realizadas ao longo das disciplinas
        \end{enumerate}
    \end{enumerate}
    \item Pré-condição: Atores devem estar autenticados e grupo de trabalho deve estar cadastrado no sistema
\end{enumerate}


% ========== Anexos (opcional) ==========
\anexo

\end{document}
